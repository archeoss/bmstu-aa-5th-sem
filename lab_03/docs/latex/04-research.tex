\chapter{Исследовательская часть}
Ниже приведены технические характеристики устройства, на котором было проведено тестирование ПО:

\begin{itemize}
    \item Операционная система: Windows 11 используя Windows Subsystem for Linux 2 (WSL2) \cite{wsl2} Имитирующей Arch Linux \cite{arch} 64-bit.
    \item Оперативная память: 16 GB.
    \item Процессор: 11th Gen Intel(R) Core(TM) i5-11320H @ 3.20GHz \cite{i5}.
\end{itemize}

\section{Время выполнения алгоритмов}
Алгоритмы тестировались при помощи \("\)бенчмарков\("\) предоставляемых встренными средствами языка Go \cite{go}.
Пример такого \("\)бенчмарка\("\) приведен в листинге \ref{lst:benchmark}.
Количество повторов регулируется тестирующей системой самоятоятельно, хотя при желание оные можно задать.
Точное количество повторов определяется в зависимости от того, был ли получен стабильный результат согласно системе.

\lstinputlisting[style=go, caption={Пример бенчмарка}, label={lst:benchmark} , firstline=144, lastline=159]{../../src/algos/sorting_test.go}

Таблица времени выполнения сортировок на отсортированных данных (в наносекундах)\newline
\includesvg[%
    width=18cm,height=15cm,inkscapelatex=false,
%inkscapeformat=pdf,
%  inkscapelatex=false,
%  distort=true,
%    angle=-12.5,
%  extractdistort=false,
%  extractangle=inherit,
]{plot-sorted}%

Таблица времени выполнения сортировок на отсортированных данных в обратном порядке (в наносекундах)\newline
\includesvg[%
    width=18cm,height=15cm,inkscapelatex=false,
%inkscapeformat=pdf,
%  inkscapelatex=false,
%  distort=true,
%    angle=-12.5,
%  extractdistort=false,
%  extractangle=inherit,
]{plot-reversed}%

Таблица времени выполнения сортировок на случайных данных (в наносекундах)\newline
    \includesvg[%
        width=18cm,height=15cm,inkscapelatex=false,
        %inkscapeformat=pdf,
%  inkscapelatex=false,
%  distort=true,
%    angle=-12.5,
%  extractdistort=false,
%  extractangle=inherit,
    ]{plot-random}%


\section{Вывод}

Как было и ожидаемо, сортировка подсчетом оказалась самой быстрой, а сортировка пузырьком самой медленной.
Быстрая сортировка оказалась быстрой, но при отсортированных данных она оказалась медленнее сортировки пузырьком.
