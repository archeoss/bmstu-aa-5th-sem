\chapter{Технологическая часть}
В данном разделе приведены требования к программному обеспечению, средства реализации и сами реализации алгоритмов.

\section{Требования к программному обеспечению}
К программе предъявляется ряд условий:
\begin{itemize}
    \item[$-$] на вход программе подается размер массива (целое число) и массив (целые числа);
    \item[$-$] на выходе программа должна выдать отсортированный массив одним из 3 способов;
    \item[$-$] при проведении эксперимента результатом работы программа
    выводит результат в текстовом виде, а также строит графики;
\end{itemize}

\section{Средства реализации}
Для реализации данной лабораторной работы необходимо установить следующее программное обеспечение:
\begin{itemize}
    \item \href{https://go.dev/}{Golang 1.19.1} - язык программирования
    \item \href{https://github.com/cep21/benchdraw}{Benchdraw 0.1.0} - Средство визуализации данных
    \item \href{https://www.latex-project.org/}{LaTeX} - система документооборота
\end{itemize}
\newpage

\section{Рализация алгоритмов}
В листингах \ref{lst:bubble-sort}, \ref{lst:counting-sort} и \ref{lst:quick-sort} приведены реализации алгоритмов
сортировки пузырьком, подсчетом и итеративная быстрая соответственно.
%\subsection{Сортировка пузырьком}
\lstinputlisting[label=lst:bubble-sort, caption=Сортировка массива пузырьком, language=go, style=go, firstline=75, lastline=84]{../../src/algos/sorting.go}
\newpage
%\subsection{Сортировка вставками}
\lstinputlisting[label=lst:counting-sort, caption=Сортировка массива подсчетом, language=go, style=go, firstline=43, lastline=73]{../../src/algos/sorting.go}
\newpage
%\subsection{Итеративная быстрая сортировка}
\lstinputlisting[label=lst:quick-sort, caption=Итеративная быстрая сортировка массива, language=go, style=go, firstline=8, lastline=41]{../../src/algos/sorting.go}

В таблице~\ref{tbl:test} приведены тесты для функций, реализующих алгоритмы сортировки.
Все тесты пройдены успешно.

\begin{table}[h!]
    \begin{center}
        \begin{threeparttable}
            \captionsetup{justification=raggedright,singlelinecheck=off}
            \caption{\label{tbl:test}Тестирование функций}
            \begin{tabular}{|c|c|c|}
                \hline
                Входной массив & Результат & Ожидаемый результат \\
                \hline
                $[1, 2, 3, 4, 5]$ & $[1, 2, 3, 4, 5]$  & $[1, 2, 3, 4, 5]$\\\hline
                $[5, 4, 3, 2, 1]$  & $[1, 2, 3, 4, 5]$ & $[1, 2, 3, 4, 5]$\\\hline
                $[-1, -2, -3, -2, -5]$  & $[-5, -3, -2, -2, -1]$  & $[-5, -3, -2, -2, -1]$\\\hline
                $[-2, 7, 0, -1, 3]$  & $[-2, -1, 0, 3, 7]$  & $[-2, -1, 0, 3, 7]$\\\hline
                $[50]$  & $[50]$  & $[50]$\\\hline
                $[-40]$  & $[-40]$  & $[-40]$\\\hline
                Пустой массив  & Пустой массив  & Пустой массив\\
                \hline
            \end{tabular}
        \end{threeparttable}
    \end{center}
\end{table}
\newpage
\section*{Вывод}

В данном разделе был разработан исходный код трёх алгоритмов сортировки: пузырьком, вставками и быстрая сортировка.

