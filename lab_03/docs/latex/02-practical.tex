\chapter{Конструкторская часть}

\section{Разработка алгоритмов}

На рисунках \ref{img:bsort}, \ref{img:csort} и рисунках \ref{img:qsort}, \ref{img:qsort_part} приведены схемы алгоритмов сортировки пузырьком,
сортировки подсчетом и быстрой сортировки соответственно.

\includeimage
{bsort} % Имя файла без расширения (файл должен быть расположен в директории inc/img/)
{f} % Обтекание (без обтекания)
{h} % Положение рисунка (см. figure из пакета float)
{0.55\textwidth} % Ширина рисунка
{Сортирвка пузырьком} % Подпись рисунка
\clearpage

\includeimage
{csort} % Имя файла без расширения (файл должен быть расположен в директории inc/img/)
{f} % Обтекание (без обтекания)
{h} % Положение рисунка (см. figure из пакета float)
{0.75\textwidth} % Ширина рисунка
{Сортировка вставками} % Подпись рисунка
\clearpage

\includeimage
{qsort} % Имя файла без расширения (файл должен быть расположен в директории inc/img/)
{f} % Обтекание (без обтекания)
{h} % Положение рисунка (см. figure из пакета float)
{0.75\textwidth} % Ширина рисунка
{Быстрая сортировка} % Подпись рисунка
\clearpage

\includeimage
{qsort_part} % Имя файла без расширения (файл должен быть расположен в директории inc/img/)
{f} % Обтекание (без обтекания)
{h} % Положение рисунка (см. figure из пакета float)
{0.75\textwidth} % Ширина рисунка
{Быстрая сортировка, разделение массива} % Подпись рисунка
\clearpage


\section{Трудоёмкость алгоритмов}

\section{Модель вычислений}

Для последующего вычисления трудоемкости введём модель вычислений:

\begin{enumerate}
    \item Операции из списка \eqref{for:opers} имеют трудоемкость 1.
    \begin{equation}
        \label{for:opers}
        +, -, /, \%, ==, !=, <, >, <=, >=, [], ++, {-}-
    \end{equation}
    \item Трудоемкость оператора выбора if условие then A else B рассчитывается, как \eqref{for:if}.
    \begin{equation}
        \label{for:if}
        f_{if} = f_{\text{условия}} +
        \begin{cases}
            f_A, & \text{если условие выполняется,}\\
            f_B, & \text{иначе.}
        \end{cases}
    \end{equation}
    \item Трудоемкость цикла рассчитывается, как \eqref{for:for}.
    \begin{equation}
        \label{for:for}
        f_{for} = f_{\text{инициализации}} + f_{\text{сравнения}} + N(f_{\text{тела}} + f_{\text{инкремента}} + f_{\text{сравнения}})
    \end{equation}
    \item Трудоемкость вызова функции равна 0.
\end{enumerate}

\section{Трудоёмкость алгоритмов}

Пусть размер массивов во всех вычислениях обозначается как $N$.

\subsection{Алгоритм сортировки пузырьком}

Трудоёмкость алгоритма сортировки пузырьком состоит из:
\begin{itemize}
    \item трудоёмкость сравнения и инкремента внешнего цикла $i \in [1..N)$ \eqref{for:bubble_outer}:
    \begin{equation}
        \label{for:bubble_outer}
        f_{i} = 2 + 2(N - 1)
    \end{equation}
    \item суммарная трудоёмкость внутренних циклов, количество итераций которых меняется в промежутке $[1..N-1]$ \eqref{for:bubble_inner}:
    \begin{equation}
        \label{for:bubble_inner}
        f_{j} = 3(N - 1) + \frac{N \cdot (N - 1)}{2} \cdot (3 + f_{if})
    \end{equation}
    \item трудоёмкость условия во внутреннем цикле \eqref{for:bubble_if}:
    \begin{equation}
        \label{for:bubble_if}
        f_{if} = 4 + \begin{cases}
                         0, & \text{в лучшем случае}\\
                         9, & \text{в худшем случае}\\
        \end{cases}
    \end{equation}
\end{itemize}

Трудоёмкость в \textbf{лучшем} случае \eqref{for:bubble_best}:
\begin{equation}
    \label{for:bubble_best}
    f_{best} = \frac{7}{2} N^2 + \frac{3}{2} N - 3 \approx \frac{7}{2} N^2 = O(N^2)
\end{equation}

Трудоёмкость в \textbf{худшем} случае \eqref{for:bubble_worst}:
\begin{equation}
    \label{for:bubble_worst}
    f_{worst} =  8N^2 - 8N - 3 \approx 8N^2 = O(N^2)
\end{equation}

\subsection{Алгоритм сортировки подсчетом}

Трудоёмкость алгоритма сортировки подсчетом состоит из:
\begin{enumerate}
    \item нахождение максимального числа в массиве $A$.
    \item трудоёмкость инициализации массива $C$ размером $k$ нулями, где $k = max(A)$.
    \item проход по массиву $A$ и подсчет количества вхождений каждого числа в массиве $C$.
    \item проход по массиву $C$ и запись чисел в массив $A$ в соответствии с их количеством в массиве $C$.
\end{enumerate}

Вычислим каждую трудоёмкость отдельно.

\begin{enumerate}
    \item Трудоёмкость нахождения максимального числа в массиве $A$\eqref{eq:counting_max}.
    \begin{equation}
        \label{eq:counting_max}
        f_{max} = 2 + 2(N - 1) = 4N - 3
    \end{equation}
    \item Трудоёмкость инициализации массива $C$\eqref{eq:init}.
    \begin{equation}
        \label{eq:init}
        f_{C} = 2 + 2(k - 1) = 4k - 3
    \end{equation}
    \item Трудоёмкость подсчета количества вхождений каждого числа в массиве $C$\eqref{eq:counting}.
    \begin{equation}
        \label{eq:counting}
        f_{count} = 2 + 2(N - 1) = 4N - 3
    \end{equation}
    \item проход по массиву $C$ и запись чисел в массив $A$ в соответствии с их количеством в массиве $C$\eqref{eq:write}.
    \begin{equation}
        \label{eq:write}
        f_{write} = 6 + 7 * N + \begin{cases}
                            0, & \text{в лучшем случае}\\
                            3k, & \text{в худшем случае}\\
        \end{cases}
    \end{equation}
\end{enumerate}

Трудоёмкость в \textbf{лучшем} случае \eqref{eq:counting_best}:
\begin{equation}
    \label{eq:counting_best}
    f_{best} = 15N + 4k - 3 \approx 15N + 4k = O(N + k)
\end{equation}

Трудоёмкость в \textbf{худшем} случае \eqref{eq:counting_worst}:
\begin{equation}
    \label{eq:counting_worst}
    f_{worst} = 15N + 7k - 3 \approx 15N + 7k = O(N + k)
\end{equation}

\subsection{Алгоритм итеративной быстрой сортировки}

Трудоёмкость алгоритма итеративной быстрой сортировки \eqref{eq:qsort}:
\begin{equation}
    \label{eq:qsort}
    T(N) = T(J) + T(N-J) + M(N)
\end{equation}

где

\begin{enumerate}
    \item $T(N)$ - трудоемкость быстрой сортировки массива размера N.
    \item $T(J)$ - трудоемкость быстрой сортировки массива размера J.
    \item $T(N-J)$ - трудоемкость быстрой сортировки массива размера N-J.
    \item $M(N)$ - трудоёмкость разделения массива на две части.
\end{enumerate}

\textbf{Вычислим для лучшего случая:}
\begin{itemize}
    \item $T(N) = 2T(\frac{N}{2}) + C*N$ - трудоемкость быстрой сортировки массива размера N.
    \begin{itemize}
        \item $2T(\frac{N}{2})$ поскольку мы разделяем массив на 2 равные части
        \item $C*N$ поскольку мы будем проходить все элементы массива на каждом уровне "дерева"
    \end{itemize}
    \item Следующий шаг - разделить дальше на 4 части (\eqref{eq:qfirst_eq} и \eqref{eq:qfirst_eq_2}):
    \begin{equation}
        \label{eq:qfirst_eq}
        T(N) = 2(2*T(\frac{N}{4}) + C*N/2) + C*N
    \end{equation}
    \begin{equation}
        \label{eq:qfirst_eq_2}
        T(N) = 4T(\frac{N}{4}) + 2C*N
    \end{equation}
    \item В общем случае \eqref{eq:qmain_eq}:
    \begin{equation}
        \label{eq:qmain_eq}
        T(N) = 2^k * T(N/(2^k)) + k*C*N
    \end{equation}
    \item Для лучшего случая - $N = 2^k$ - идеально распределенное дерево (отсюда следует, что $k = log_2(N)$) \eqref{eq:qk_eq}
    \begin{equation}
        \label{eq:qk_eq}
        T(N) = 2^k * T(1) + k * C * (2^k)
    \end{equation}
    \item[$-$] Вычислим T(1) - трудоемкость при массиве длинной 1 - и C - трудоемкость разделения \eqref{eq:qconst}:
    \begin{equation}
        \label{eq:qconst}
        T(1) = 6 + 1 * 9 = 15;
        C = 12 + N / (2^k) * 8
    \end{equation}
    \item Полная сложность (при $k = log_2(N)$) \eqref{eq:qsort_best}:
    \begin{equation}
        \label{eq:qsort_best}
        T(N) = 15N + log_2(N)*(12 + 8) = 15N + 20Nlog_2(N)
    \end{equation}
\end{itemize}

\textbf{Теперь вычислим для худшего случая:}
\begin{itemize}
    \item $T(N) = T(N - 1) + C*N$ - трудоемкость быстрой сортировки массива размера N.
    \begin{itemize}
        \item $T(N - 1)$ поскольку мы разделяем массив на 2 неравные части: пустое множество и полное множество за исключением "середины"
    \end{itemize}
    \item Следующие шаги очевидны \eqref{eq:qwfirst_eq}, \eqref{eq:qwfirst_eq_2}:
    \begin{equation}
        \label{eq:qwfirst_eq}
        T(N) = T(N-2) + C(N-1) + C*N =  T(N-2) + 2C*N - C
    \end{equation}
    \begin{equation}
        \label{eq:qwfirst_eq_2}
        T(N) = T(N-3) + 3C*N - 2C*N - C
    \end{equation}
    \item В общем случае \eqref{eq:qwmain_eq}:
    \begin{equation}
        \label{eq:qwmain_eq}
        T(N) = T(N-k) + k*C*N - C(\frac{k(k-1)}{2})
    \end{equation}
    \item Для худшего случая - $N = k$ - нераспределенное дерево \eqref{eq:qwk_eq}:
    \begin{equation}
        \label{eq:qwk_eq}
        T(N) = T(0) + N*C*N - C(\frac{N(N-1)}{2})
    \end{equation}
    \item[$-$] Вычислим параметры \eqref{eq:qwconst}:
    \begin{equation}
        \label{eq:qwconst}
        T(0) = 1;
        C = 12 + (N-k) * 8
    \end{equation}
    \item Полная сложность (при $N = k$) \eqref{eq:qsort_worst}:
    \begin{equation}
        \label{eq:qsort_worst}
        T(N) = 1 + N*N*12 - 12*(\frac{N(N-1)}{2}) = 1 + 6N^2 + 6N
    \end{equation}
\end{itemize}

Трудоёмкость в \textbf{лучшем} случае \eqref{eq:qsort_best_fin}:
\begin{equation}
    \label{eq:qsort_best_fin}
    f_{best} = 15N + 20Nlog_2(N) \approx 20Nlog_2(N) = O(Nlog_2(N))
\end{equation}

Трудоёмкость в \textbf{худшем} случае \eqref{eq:qsort_worst_fin}:
\begin{equation}
    \label{eq:qsort_worst_fin}
    f_{worst} = 1 + 6N^2 + 6N \approx 6N^2 = O(N^{2})
\end{equation}
\section*{Вывод}

На основе теоретических данных, полученных из аналитического раздела, были построены схемы трёх алгоритмов сортировки.
Оценены их трудоёмкости в лучшем и худшем случаях.

\clearpage
