\chapter*{ЗАКЛЮЧЕНИЕ}
\addcontentsline{toc}{chapter}{ЗАКЛЮЧЕНИЕ}

В ходе выполнения лабораторной работы были достигнуты следующие задачи:

\begin{itemize}
    \item были изучены и реализованы 3 алгоритма сортировки: пузырёк, подсчетом, быстрая сортировка;
    \item были приведены аналитические данные о выше перечисленных алгоритмах;
    \item были приведены подробные блок-схемы, описывающие алгоритмы;
    \item был проведён сравнительный анализ алгоритмов на основе экспериментальных данных.
\end{itemize}

Экспериментальные данные показали различные сильные и слабые стороны каждого алгоритма.
Так например:
\begin{itemize}
    \item[$-$] сортировка пузырьком работает крайне медленно независимо от входных данных;
    \item[$-$] быстрая сортировка работает быстрее на случайных данных, поэтому лучше всего подходит как общее решение абстрактной задачи на сортировку данных;
    \item[$-$] быстрая сортировка работает медленно на отсортированном массиве.
    \item[$-$] сортировка подсчетом работает быстрее всех выше перечисленных алгоритмов, однако требует сильно больше памяти, чем остальные алгоритмы.
\end{itemize}

\indentЦели лабораторной работы по изучению и исследованию особенностей алгоритмов сортировок были достигнуты