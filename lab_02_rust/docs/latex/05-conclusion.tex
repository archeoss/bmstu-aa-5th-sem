\chapter*{ЗАКЛЮЧЕНИЕ}
\addcontentsline{toc}{chapter}{ЗАКЛЮЧЕНИЕ}

В рамках данной лабораторной работы:

\begin{enumerate}
    \item Были изучены и реализованы 3 алгоритма перемножения матриц: обычный, Копперсмита$-$Винограда, оптимизированный Копперсмита$-$Винограда;
    \item Был произведён анализ трудоёмкости алгоритмов на основе теоретических расчётов и выбранной модели вычислений;
    \item Был сделан сравнительный анализ алгоритмов на основе экспериментальных данных.
\end{enumerate}

% TODO: циферки
На основании анализа трудоёмкости алгоритмов в выбранной модели вычислений было показано, что улучшенный алгоритм Винограда имеет меньшую сложность, нежели простой алгоритм перемножения матриц. На основании замеров времени исполнения алгоритмов, был сделан вывод, что алгоритм Копперсмита -- Винограда в среднем в 3.5 раза быстрее чем обычный алгоритм умножения матриц. Кроме этого, я решил добавить в сравнение алгоритм с предварительным транспонированием матрицы. Оказалось, что такая реализация быстрее алгоритма Копперсмита -- Винограда в 1.5 раза и обгоняет классический алгоритм умножения в 3.6 раза.
