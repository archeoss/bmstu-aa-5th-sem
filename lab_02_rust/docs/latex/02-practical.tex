\chapter{Конструкторская часть}

\section{Разработка алгоритмов}

На рисунке \ref{img:naive_mut} приведена схема стандартного алгоритма умножения матриц.

На рисунках \ref{img:winograd_mut0}, \ref{img:winograd_mut01}, \ref{img:winograd_mut10} и \ref{img:winograd_mut11} представлена схема алгоритма Копперсмита$-$Винограда.

На рисунках \ref{img:winograd_imp_mut0}, \ref{img:winograd_imp_mut01}, \ref{img:winograd_imp_mut10} и \ref{img:winograd_imp_mut11} представлена схема оптимизированного алгоритма Копперсмита$-$Винограда.

Для алгоритма Копперсмита$-$Винограда худшим случаем являются матрицы с нечётным общим размером, а лучшим - с чётным,
из-за того что отпадает необходимость в последнем цикле.

\includeimage
{naive_mut} % Имя файла без расширения (файл должен быть расположен в директории inc/img/)
{f} % Обтекание (без обтекания)
{h} % Положение рисунка (см. figure из пакета float)
{0.75\textwidth} % Ширина рисунка
{Классическое перемножение матриц} % Подпись рисунка
\clearpage

\includeimage
{winograd_mut0} % Имя файла без расширения (файл должен быть расположен в директории inc/img/)
{f} % Обтекание (без обтекания)
{h} % Положение рисунка (см. figure из пакета float)
{0.65\textwidth} % Ширина рисунка
{Перемножение матриц с помощью алгоритма Копперсмита$-$Винограда, часть 1} % Подпись рисунка
\clearpage

\includeimage
{winograd_mut01} % Имя файла без расширения (файл должен быть расположен в директории inc/img/)
{f} % Обтекание (без обтекания)
{h} % Положение рисунка (см. figure из пакета float)
{0.65\textwidth} % Ширина рисунка
{Перемножение матриц с помощью алгоритма Копперсмита$-$Винограда, часть 2} % Подпись рисунка
\clearpage

\includeimage
{winograd_mut10} % Имя файла без расширения (файл должен быть расположен в директории inc/img/)
{f} % Обтекание (без обтекания)
{h} % Положение рисунка (см. figure из пакета float)
{0.75\textwidth} % Ширина рисунка
{Перемножение матриц с помощью алгоритма Копперсмита$-$Винограда, вычисление векторов, часть 1} % Подпись рисунка
\clearpage

\includeimage
{winograd_mut11} % Имя файла без расширения (файл должен быть расположен в директории inc/img/)
{f} % Обтекание (без обтекания)
{h} % Положение рисунка (см. figure из пакета float)
{0.75\textwidth} % Ширина рисунка
{Перемножение матриц с помощью алгоритма Копперсмита$-$Винограда, вычисление векторов, часть 2} % Подпись рисунка
\clearpage

\includeimage
{winograd_imp_mut0} % Имя файла без расширения (файл должен быть расположен в директории inc/img/)
{f} % Обтекание (без обтекания)
{h} % Положение рисунка (см. figure из пакета float)
{0.65\textwidth} % Ширина рисунка
{Оптимизированное перемножение матриц с помощью алгоритма Копперсмита$-$Винограда, часть 1} % Подпись рисунка
\clearpage

\includeimage
{winograd_imp_mut01} % Имя файла без расширения (файл должен быть расположен в директории inc/img/)
{f} % Обтекание (без обтекания)
{h} % Положение рисунка (см. figure из пакета float)
{0.65\textwidth} % Ширина рисунка
{Оптимизированное перемножение матриц с помощью алгоритма Копперсмита$-$Винограда, часть 2} % Подпись рисунка
\clearpage

\includeimage
{winograd_imp_mut10} % Имя файла без расширения (файл должен быть расположен в директории inc/img/)
{f} % Обтекание (без обтекания)
{h} % Положение рисунка (см. figure из пакета float)
{0.75\textwidth} % Ширина рисунка
{Оптимизированное перемножение матриц с помощью алгоритма Копперсмита$-$Винограда, вычисление векторов, часть 1} % Подпись рисунка
\clearpage

\includeimage
{winograd_imp_mut11} % Имя файла без расширения (файл должен быть расположен в директории inc/img/)
{f} % Обтекание (без обтекания)
{h} % Положение рисунка (см. figure из пакета float)
{0.75\textwidth} % Ширина рисунка
{Оптимизированное перемножение матриц с помощью алгоритма Копперсмита$-$Винограда, вычисление векторов, часть 2} % Подпись рисунка
\clearpage

\section{Модель вычислений}

Для последующего вычисления трудоемкости введём модель вычислений:

\begin{enumerate}
    \item Операции из списка \eqref{eq:opers} имеют трудоемкость 1.
    \begin{equation}
        \label{eq:opers}
        +, -, \%, ==, !=, <, >, <=, >=, [], ++, {-}-
    \end{equation}
    \item Операции из списка \eqref{eq:opers2} имеют трудоемкость 2.
    \begin{equation}
        \label{eq:opers2}
        *, /
    \end{equation}
    \item Трудоемкость оператора выбора if условие then A else B рассчитывается, как \eqref{eq:if}.
    \begin{equation}
        \label{eq:if}
        f_{if} = f_{\text{условия}} +
        \begin{cases}
            f_A, & \text{если условие выполняется,}\\
            f_B, & \text{иначе.}
        \end{cases}
    \end{equation}
    \item Трудоемкость цикла рассчитывается, как \eqref{eq:for}.
    \begin{equation}
        \label{eq:for}
        f_{for} = f_{\text{инициализации}} + f_{\text{сравнения}} + N(f_{\text{тела}} + f_{\text{инкремента}} + f_{\text{сравнения}})
    \end{equation}
    \item Трудоемкость вызова функции равна 0.
\end{enumerate}

\section{Трудоёмкость алгоритмов}

\subsection{Стандартный алгоритм умножения матриц}

Трудоёмкость стандартного алгоритма умножения матриц состоит из:
\begin{itemize}
    \item Внешнего цикла по $i \in [1..A]$, трудоёмкость которого: $f = 2 + A \cdot (2 + f_{body})$;
    \item Цикла по $j \in [1..C]$, трудоёмкость которого: $f = 2 + C \cdot (2 + f_{body})$;
    \item Скалярного умножения двух векторов - цикл по $k \in [1..B]$, трудоёмкость которого: $f = 2 + 10B$.
\end{itemize}

Трудоёмкость стандартного алгоритма равна трудоёмкости внешнего цикла, можно вычислить ее, подставив циклы тела \eqref{eq:base}.
\begin{equation}
    \label{eq:base}
    f_{base} = 2 + A \cdot (4 + C \cdot (4 + 10B)) = 2 + 4A + 4AC + 10ABC \approx 10ABC
\end{equation}

\subsection{Алгоритм Копперсмита — Винограда}

Трудоёмкость алгоритма Копперсмита — Винограда состоит из:

\begin{enumerate}
    \item Создания векторов rows и cols \eqref{eq:init}.
    \begin{equation}
        \label{eq:init}
        f_{create} = A + C;
    \end{equation}

    \item Заполнения вектора rows \eqref{eq:MH}.
    \begin{equation}
        \label{eq:MH}
        f_{rows} = 3 + \frac{B}{2} \cdot (5 + 12A);
    \end{equation}

    \item Заполнения вектора cols \eqref{eq:MV}.
    \begin{equation}
        \label{eq:MV}
        f_{cols} = 3 + \frac{B}{2} \cdot (5 + 12C);
    \end{equation}

    \item Цикла заполнения матрицы для чётных размеров \eqref{eq:cycle}.
    \begin{equation}
        \label{eq:cycle}
        f_{cycle} = 2 + A \cdot (4 + C \cdot (11 + \frac{25}{2} \cdot B));
    \end{equation}

    \item Цикла, для дополнения умножения суммой последних нечётных строки и столбца, если общий размер нечётный \eqref{eq:last}.
    \begin{equation}
        \label{eq:last}
        f_{last} = \begin{cases}
                       2, & \text{чётная,}\\
                       4 + A \cdot (4 + 14C), & \text{иначе.}
        \end{cases}
    \end{equation}
\end{enumerate}

Итого, для худшего случая (нечётный размер матриц):
\begin{equation}
    \label{eq:bad}
    f_{wino\_w} = A + C + 12 + 8A + 5B + 6AB + 6CB + 25AC + \frac{25}{2}ABC \approx 12.5 \cdot MNK
\end{equation}

Для лучшего случая (чётный размер матриц):
\begin{equation}
    \label{eq:good}
    f_{wino\_b} = A + C + 10 + 4A + 5B + 6AB + 6CB + 11AC + \frac{25}{2}ABC \approx 12.5 \cdot MNK
\end{equation}

\subsection{Оптимизированный алгоритм Копперсмита — Винограда}

Трудоёмкость улучшенного алгоритма Копперсмита — Винограда состоит из:
\begin{enumerate}
    \item Создания векторов rows и cols \eqref{eq:impr_init}.
    \begin{equation}
        \label{eq:impr_init}
        f_{init} = A + C;
    \end{equation}

    \item Заполнения вектора rows \eqref{eq:impr_MH}.
    \begin{equation}
        \label{eq:impr_MH}
        f_{rows} = 2 + \frac{B}{2} \cdot (4 + 8A);
    \end{equation}

    \item Заполнения вектора cols \eqref{eq:impr_MV}.
    \begin{equation}
        \label{eq:impr_MV}
        f_{cols} = 2 + \frac{B}{2} \cdot (4 + 8A);
    \end{equation}

    \item Цикла заполнения матрицы для чётных размеров \eqref{eq:impr_cycle}.
    \begin{equation}
        \label{eq:impr_cycle}
        f_{cycle} = 2 + A \cdot (4 + C \cdot (8 + 9B))
    \end{equation}

    \item Цикла, для дополнения умножения суммой последних нечётных строки и столбца, если общий размер нечётный \eqref{eq:impr_last}.
    \begin{equation}
        \label{eq:impr_last}
        f_{last} =
        \begin{cases}
            2, & \text{чётная,}\\
            4 + A \cdot (4 + 12C), & \text{иначе.}
        \end{cases}
    \end{equation}
\end{enumerate}

Итого, для худшего случая (нечётный общий размер матриц) имеем \eqref{eq:bad_impr}.
\begin{equation}
    \label{eq:bad_impr}
    f = A + C + 10 + 4B + 4BC + 4BA + 8A + 20AC + 9ABC \approx 9ABC
\end{equation}

Для лучшего случая (чётный общий размер матриц) имеем \eqref{eq:good_impr}.
\begin{equation}
    \label{eq:good_impr}
    f = A + C + 8 + 4B +4BC + 4BA + 4A + 8AC + 9ABC \approx 9ABC
\end{equation}

\section*{Вывод}
На основе теоретических данных, полученных из аналитического раздела, были построены схемы обоих алгоритмов умножения матриц.
Оценены их трудоёмкости в лучшем и худшем случаях.
