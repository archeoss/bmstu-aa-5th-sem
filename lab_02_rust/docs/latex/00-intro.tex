\chapter*{ВВЕДЕНИЕ}
\addcontentsline{toc}{chapter}{ВВЕДЕНИЕ}

\textbf{Алгоритм Копперсмита$-$Винограда}\cite{winograd} $-$ алгоритм умножения квадратных матриц, предложенный в 1987 году Д. Копперсмитом и Ш. Виноградом.
В исходной версии асимптотическая сложность алгоритма составляла $O(n^{2,3755})$, где  $n$ — размер стороны матрицы.
На текущий момент сложнотсь алгоритма составляет $O(n^{2,3728596})$.

\textbf{Цель лабораторной работы:}
Изучение и исследование особенностей оптимизации сложных вычислений.

\textbf{Задачи лабораторной работы:}
\begin{enumerate}
\item Изучить и реализовать алгоритмы перемножения матриц:
\begin{enumerate}
    \item[$-$] Классический;
    \item[$-$] Копперсмита$-$Винограда;
    \item[$-$] Копперсмита$-$Винограда с оптимизациями согласно варианту.
\end{enumerate}
\item Создать программное обеспечение, реализующее алгоритмы, указанные в варианте.
\item Провести анализ затрат работы программы по времени и по памяти, выяснить влияющие на них характеристики.
    \item Создать отчёт, содержащий:
    \begin{enumerate}
        \item[$-$] Актуальность исследования;
        \item[$-$] Xарактеристики предложенной реализации (по времени и памяти);
        \item[$-$] Краткие рекомендации об особенностях применения оптимизаций (важно помнить об улучшениях, которые использует компилятор);
        \item[$-$] Результаты тестирования;
        \item[$-$] Выводы.
    \end{enumerate}
\end{enumerate}
\newpage
\textbf{Для достижения поставленных целей и задач необходимо:}
\begin{enumerate}
    \item Изучить теоретические основы алгоритма Копперсмита$-$Винограда;
    \item Реализовать выше обозначенный алгоритм;
    \item Изучить и реализовать возможные пути оптимизации;
    \item Провести экспериментальное исследование.
\end{enumerate}

\textbf{В ходе работы будут затронуты следующие темы:}
\begin{enumerate}
    \item Оптимизация вычислений;
    \item Алгоритмы перемножения матриц;
    \item Оценка реализаций алгоритмов.
\end{enumerate}
