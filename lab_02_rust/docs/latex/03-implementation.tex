\chapter{Технологическая часть}

В данном разделе приведены требования к программному обеспечению, средства реализации и листинги кода.

\section{Требования к программному обеспечению}
К программе предъявляется ряд условий:
\begin{itemize}
    \item[$-$] На вход подается 3 числа (A, B, C), определяющие размеры матриц, а также сами матрицы размерами AxB и BxC;
    \item[$-$] На выход ПО должно выводить результат перемножения 2 матриц;
    \item[$-$] ПО должно быть написано на языке Rust;
\end{itemize}

\section{Средства реализации}
Для реализации данной лабораторной работы необходимо установить следующее программное обеспечение:
\begin{itemize}
    \item \href{https://www.rust-lang.org/}{Rust Programming Language v1.64.0} - язык программирования
    \item \href{https://github.com/bheisler/criterion.rs}{Criterion.rs v0.4.0} - Средство визуализации данных
    \item \href{https://www.latex-project.org/}{LaTeX} - система документооборота
\end{itemize}

\section{Реализация алгоритмов}
В следующих листингах представлены следующие алгоритмы:
\begin{itemize}
    \item[1.] В листинге \ref{lst:naive-mut} представлен классический алгоритм перемножения матриц
    \item[2.] В листингах \ref{lst:winograd0} и \ref{lst:winograd1} представлены алгоритм перемножения матриц с использованием алгоритма Копперсмита$-$Винограда
    \item[3.] В листингах \ref{lst:winograd_imp0} и \ref{lst:winograd_imp1} представлены алгоритм перемножения матриц с использованием оптимизированного алгоритма Копперсмита$-$Винограда
\end{itemize}
\newpage

\subsection{Классический алгоритм перемножения матриц}
\lstinputlisting[caption={Классический алгоритм перемножения матриц}, label={lst:naive-mut}, language=Rust, style=rust,
firstline=73, lastline=81]{../../src/matrix/algos.rs}

Замечание: здесь и далее под ключевым словом self понимается ссылка на экземпляр типа матрицы ($Vec<Vec<i32>>$).
\newpage

\subsection{Aлгоритм перемножения матриц Копперсмита$-$Винограда}

\lstinputlisting[caption={Алгоритм перемножения матриц Копперсмита$-$Винограда}, label={lst:winograd0}, language=Rust, style=rust,
firstline=86, lastline=116]{../../src/matrix/algos.rs}
\newpage
\lstinputlisting[caption={Алгоритм перемножения матриц Копперсмита$-$Винограда, вычисление векторов}, label={lst:winograd1}, language=Rust, style=rust,
firstline=18, lastline=43]{../../src/matrix/algos.rs}
\newpage
\subsection{Оптимизированный алгоритм перемножения матриц Копперсмита$-$Винограда}
\lstinputlisting[caption={Оптимизированный алгоритм перемножения матриц Копперсмита$-$Винограда}, label={lst:winograd_imp0}, language=Rust, style=rust,
firstline=118, lastline=149]{../../src/matrix/algos.rs}
\newpage
\lstinputlisting[caption={Оптимизированный алгоритм перемножения матриц Копперсмита$-$Винограда, вычисление векторов}, label={lst:winograd_imp1}, language=Rust, style=rust,
firstline=45, lastline=69]{../../src/matrix/algos.rs}
\newpage
\section{Тестовые данные}

В таблице~\ref{tabular:test_rec} приведены тесты для функций, реализующих стандартный алгоритм умножения матриц, алгоритм Винограда и оптимизированный алгоритм Винограда. Все тесты пройдены успешно.

\begin{table}[h!]
    \begin{center}
        \begin{threeparttable}
            \captionsetup{justification=raggedright,singlelinecheck=off}
            \caption{\label{tabular:test_rec} Тестирование функций}
            \begin{tabular}{|c|c|c|c|c|c|}
                \hline
                Первая матрица & Вторая матрица & Ожидаемый результат \\ \hline
    %            \vspace{4mm}
                $\begin{pmatrix}
                     1 & 2 & 3\\
                     1 & 2 & 3\\
                     1 & 2 & 3
                \end{pmatrix}$ &
                $\begin{pmatrix}
                     1 & 2 & 3\\
                     1 & 2 & 3\\
                     1 & 2 & 3
                \end{pmatrix}$ &
                $\begin{pmatrix}
                     6 & 12 & 18\\
                     6 & 12 & 18\\
                     6 & 12 & 18
                \end{pmatrix}$ \\
    %            \vspace{2mm}
                \hline
    %            \vspace{2mm}
                $\begin{pmatrix}
                     1 & 2 & 2\\
                     1 & 2 & 2
                \end{pmatrix}$ &
                $\begin{pmatrix}
                     1 & 2\\
                     1 & 2\\
                     1 & 2
                \end{pmatrix}$ &
                $\begin{pmatrix}
                     5 & 10\\
                     5 & 10
                \end{pmatrix}$ \\
    %            \vspace{2mm}
                \hline
    %            \vspace{2mm}
                $\begin{pmatrix}
                     2
                \end{pmatrix}$ &
                $\begin{pmatrix}
                     2
                \end{pmatrix}$ &
                $\begin{pmatrix}
                     4
                \end{pmatrix}$ \\
    %            \vspace{2mm}
                \hline
    %            \vspace{2mm}
                $\begin{pmatrix}
                     1 & -2 & 3\\
                     1 & 2 & 3\\
                     1 & 2 & 3
                \end{pmatrix}$ &
                $\begin{pmatrix}
                     -1 & 2 & 3\\
                     1 & 2 & 3\\
                     1 & 2 & 3
                \end{pmatrix}$ &
                $\begin{pmatrix}
                     0 & 4 & 6\\
                     4 & 12 & 18\\
                     4 & 12 & 18
                \end{pmatrix}$\\
    %            \vspace{2mm}
                \hline
    %            \vspace{2mm}
            \end{tabular}
        \end{threeparttable}
    \end{center}
\end{table}

\section*{Вывод}

В данном разделе были разработаны исходные коды четырёх алгоритмов перемножения матриц: обычный алгоритм, алгоритм с транспонированием, алгоритм Копперсмита — Винограда, оптимизированный алгоритм Копперсмита — Винограда.
