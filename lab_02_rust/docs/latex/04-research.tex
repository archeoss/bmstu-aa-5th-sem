\chapter{Исследовательская часть}
Ниже приведены технические характеристики устройства, на котором было проведено тестирование ПО:

\begin{itemize}
    \item Операционная система: Arch Linux \cite{arch} 64-bit.
    \item Оперативная память: 16 Гб.
    \item Процессор: 11th Gen Intel\textsuperscript{\tiny\textregistered} Core\textsuperscript{\tiny\texttrademark} i5-11320H @ 3.20 ГГц\cite{i5}.
\end{itemize}

\section{Пример выполнения}
\includeimage
{example} % Имя файла без расширения (файл должен быть расположен в директории inc/img/)
{f} % Обтекание (без обтекания)
{h} % Положение рисунка (см. figure из пакета float)
{0.75\textwidth} % Ширина рисунка
{Пример выполнения программы} % Подпись рисунка
\clearpage

\section{Время выполнения алгоритмов}
Алгоритмы тестировались при помощи инструментов замера времени предоставляемых библиотекой Criterion.rs\cite{Criterion}.
Пример функции по замеру времени приведен в листинге \ref{lst:benchmark}.
Количество повторов регулируется тестирующей системой самостоятельно, однако ввиду трудоемкости вычислений, количество повторов было ограничено до 12.
\lstinputlisting[language=Rust, style=rust, caption={Пример функции замера времени}, label={lst:benchmark} , firstline=7, lastline=30]{../../benches/bench.rs}

\newpage
График, показывающий время перемножения матриц 3 методами (четный размер матриц)\newline
%\includesvg[%
%    width=18cm,height=15cm,inkscapelatex=false,
%%inkscapeformat=pdf,
%%  inkscapelatex=false,
%%  distort=true,
%%    angle=-12.5,
%%  extractdistort=false,
%%  extractangle=inherit,
%]{even-bench.svg}%
\includeimage
{even_bench} % Имя файла без расширения (файл должен быть расположен в директории inc/img/)
{f} % Обтекание (без обтекания)
{h} % Положение рисунка (см. figure из пакета float)
{1\textwidth} % Ширина рисунка
{График зависимости времени выполнеия от размера выполнения, четный размер} % Подпись рисунка
\newpage

График, показывающий время перемножения матриц 3 методами (нечетный размер матриц)\newline
%\includesvg[%
%    width=18cm,height=15cm,inkscapelatex=false,
%%inkscapeformat=pdf,
%%  inkscapelatex=false,
%%  distort=true,
%%    angle=-12.5,
%%  extractdistort=false,
%%  extractangle=inherit,
%]{odd-bench.svg}%
\includeimage
{odd_bench} % Имя файла без расширения (файл должен быть расположен в директории inc/img/)
{f} % Обтекание (без обтекания)
{h} % Положение рисунка (см. figure из пакета float)
{1\textwidth} % Ширина рисунка
{График зависимости времени выполнеия от размера выполнения, нечетный размер} % Подпись рисунка

\section*{Вывод}

Время работы алгоритма Винограда незначительно меньше стандартного алгоритма умножения, однако при должной оптимизации есть возможность сильно лучше время выполнения.
