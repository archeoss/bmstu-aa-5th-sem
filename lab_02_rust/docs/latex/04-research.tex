\chapter{Исследовательская часть}
Ниже приведены технические характеристики устройства, на котором было проведено тестирование ПО:

\begin{itemize}
    \item Операционная система: Arch Linux \cite{arch} 64-bit.
    \item Оперативная память: 16 Гб.
    \item Процессор: 11th Gen Intel^R Core^{TM} i5-11320H @ 3.20 ГГц \cite{i5}.
\end{itemize}

\section{Пример выполнения}
\includesvg[%
    width=18cm,height=15cm,inkscapelatex=false,
%inkscapeformat=pdf,
%  inkscapelatex=false,
%  distort=true,
%    angle=-12.5,
%  extractdistort=false,
%  extractangle=inherit,
]{example}%
\newpage

\section{Время выполнения алгоритмов}
Алгcaоритмы тестировались при помощи инструментов замера времени предоставляемых библиотекой Criterion.rs \cite{Criterion}.
Пример функции по замеру времени приведен в листинге \ref{lst:benchmark}.
Количество повторов регулируется тестирующей системой самостоятельно, однако ввиду трудоемкости вычислений, количество повторов было ограничено до 12.
\lstinputlisting[language=Rust, style=rust, caption={Пример бенчмарка}, label={lst:benchmark} , firstline=7, lastline=30]{../../benches/bench.rs}
% TODO: Графики, четные, нечетные

График, показывающий время перемножения матриц 3 методами (четный размер матриц)\newline
\includesvg[%
    width=18cm,height=15cm,inkscapelatex=false,
%inkscapeformat=pdf,
%  inkscapelatex=false,
%  distort=true,
%    angle=-12.5,
%  extractdistort=false,
%  extractangle=inherit,
]{even-bench}%

График, показывающий время перемножения матриц 3 методами (нечетный размер матриц)\newline
\includesvg[%
    width=18cm,height=15cm,inkscapelatex=false,
%inkscapeformat=pdf,
%  inkscapelatex=false,
%  distort=true,
%    angle=-12.5,
%  extractdistort=false,
%  extractangle=inherit,
]{odd_bench}%

% TODO: циферки
\section*{Вывод}

Время работы алгоритма Винограда незначительно меньше стандартного алгоритма умножения, однако при должной оптимизации есть возможность сильно лучше время выполнения.