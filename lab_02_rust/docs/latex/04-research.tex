\chapter{Исследовательская часть}
Ниже приведены технические характеристики устройства, на котором было проведено тестирование ПО:

\begin{itemize}
    \item Операционная система: Windows 11 используя Windows Subsystem for Linux 2 (WSL2) \cite{wsl2} Имитирующей Arch Linux \cite{arch} 64-bit.
    \item Оперативная память: 16 Гб.
    \item Процессор: 11th Gen Intel(R) Core(TM) i5-11320H @ 3.20 ГГц \cite{i5}.
\end{itemize}

\section{Время выполнения алгоритмов}
Алгоритмы тестировались при помощи \("\)бенчмарков\("\) предоставляемых библиотекой Criterion.rs \cite{Criterion}.
Пример такого \("\)бенчмарка\("\) приведен в листинге \ref{lst:benchmark}.
Количество повторов обычно регулируется тестирующей системой самостоятельно, однако ввиду трудоемкости вычислений, количество повторов было ограничено до 20.
\lstinputlisting[language=Rust, style=rust, caption={Пример бенчмарка}, label={lst:benchmark} , firstline=7, lastline=30]{../../benches/bench.rs}
% TODO: Графики, четные, нечетные
%График времени выполнения алгоритмов на случайных данных (в секундах)\newline
%\includesvg[%
%    width=18cm,height=15cm,inkscapelatex=false,
%%inkscapeformat=pdf,
%%  inkscapelatex=false,
%%  distort=true,
%%    angle=-12.5,
%%  extractdistort=false,
%%  extractangle=inherit,
%]{plot-time}%

%График времени выполнения алгоритмов на четном данных (в секундах)\newline
%\includesvg[%
%    width=18cm,height=15cm,inkscapelatex=false,
%%inkscapeformat=pdf,
%%  inkscapelatex=false,
%%  distort=true,
%%    angle=-12.5,
%%  extractdistort=false,
%%  extractangle=inherit,
%]{plot-time}%

% TODO: циферки
\section{Вывод}

Реализация умножения матриц с помощью алгоритма Копперсмита$-$Винограда в среднем выполняется в 3.4 раза быстрее, чем умножение обычным способом. Но, не смотря на это, алгоритм матриц с предварительным транспонированием матрицы выполняется быстрее чем любой рассматриваемый алгоритм. Такой алгоритм выполняется быстрее 3.6 раза чем обычное умножение, в 1.5 раза быстрее чем алгоритм Копперсмита$-$Винограда и в 1.3 раза быстрее чем его оптимизированная версия. Можно сделать вывод, что это происходит из-за кэширования соседних элементов перемножаемых строк. Кроме этого, для такого алгоритма не нужна дополнительная память, и, следовательно, нет нужды в дополнительном времени для ее выделения.
