\chapter*{ВВЕДЕНИЕ}
\addcontentsline{toc}{chapter}{ВВЕДЕНИЕ}

\textbf{Многопоточность} $-$ способность центрального процессора (CPU) или одного ядра в многоядерном процессоре одновременно выполнять несколько процессов или потоков, соответствующим образом поддерживаемых операционной системой.

Этот подход отличается от многопроцессорности, так как многопоточность процессов и потоков совместно использует ресурсы одного или нескольких ядер: вычислительных блоков, кэш-памяти ЦПУ или буфера перевода с преобразование.

Для распараллеливания может быть рассмотрена задача кластеризации пространственных данных. 

Алгоритм DBSCAN (Density Based Spatial Clustering of Applications with Noise), был предложен Мартином Эстер, Гансом-Питером Кригель и коллегами в 1996 году как решение проблемы разбиения (изначально пространственных) данных на кластеры произвольной формы. 
Большинство алгоритмов, производящих плоское разбиение, создают кластеры по форме близкие к сферическим, так как минимизируют расстояние документов до центра кластера.
Авторы DBSCAN экспериментально показали, что их алгоритм способен распознать кластеры различной формы.
\clearpage
\textbf{Цель лабораторной работы:}
\begin{itemize}
    \item[$-$] Изучение и реализация параллельных вычислений
\end{itemize}

\textbf{Задачи лабораторной работы:}
\begin{enumerate}
    \item Проанализировать последовательный и параллельный варианты алгоритма DBSCAN,
    \item Определить средства программной реализации алгоритмом,
    \item Реализовать разработанные алгоритмы,
    \item Провести анализ затрат работы программы по времени, выяснить влияющие на них характеристики,
    \item Создать отчёт, содержащий:
    \begin{enumerate}
        \item[$-$] актуальность исследования;
        \item[$-$] характеристики предложенной реализации (по времени);
        \item[$-$] результаты тестирования;
        \item[$-$] Выводы.
    \end{enumerate}
\end{enumerate}

\textbf{В ходе работы будут затронуты следующие темы:}
\begin{enumerate}
\item Параллелизация вычислений.
\item Алгоритм DBSCAN.
\item Многопоточность.
\end{enumerate}
