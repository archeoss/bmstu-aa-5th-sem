\chapter{Технологическая часть}

В данном разделе приведены требования к программному обеспечению, средства реализации и сами реализации алгоритмов.

\section{Требования к программному обеспечению}
К программе предъявляется ряд условий:
\begin{itemize}
    \item[$-$] На вход подается путь к файлу, содержащий некие двумерные данные, а также числа eps и MinPt, определяющие определяющие характеристики алгоритма;
    \item[$-$] На выход ПО должно выводить результат алгоритма DBSCAN в графическом виде;
    \item[$-$] ПО должно замерять время работы алгоритмов;
\end{itemize}

\section{Средства реализации}
Для реализации данной лабораторной работы необходимо установить следующее программное обеспечение:
\begin{itemize}
    \item \href{https://www.rust-lang.org/}{Rust Programming Language v1.64.0} - язык программирования
    \item \href{https://github.com/bheisler/criterion.rs}{Criterion.rs v0.4.0} - Средство визуализации данных
    \item \href{https://www.latex-project.org/}{LaTeX} - система документооборота
\end{itemize}

\section{Реализация алгоритмов}
В следующих листингах представлены следующие алгоритмы:
\begin{itemize}
    \item[1.] В листингах \ref{lst:baseSerial} и \ref{lst:expandCluster} представлен последовательный алгоритм DBSCAN.
    \item[2.] В листингах \ref{lst:baseParallel}, \ref{lst:makeCell}, \ref{lst:markCore}, \ref{lst:clusterCore} и \ref{lst:clusterBorder} представлен параллельный алгоритм DBSCAN.
\end{itemize}
\newpage

\subsection{Последовательный алгоритм DBSCAN}
\lstinputlisting[caption={Последовательный алгоритм DBSCAN}, label={lst:baseSerial}, language=Rust, style=rust,
firstline=95, lastline=122]{../../src/dbscan/model.rs}
\newpage
\lstinputlisting[caption={Последовательный алгоритм DBSCAN, расширение кластеров}, label={lst:expandCluster}, language=Rust, style=rust,
firstline=124, lastline=160]{../../src/dbscan/model.rs}
\newpage

\subsection{Параллельный алгоритм DBSCAN}

\lstinputlisting[caption={Параллельный алгоритм DBSCAN}, label={lst:baseParallel}, language=Rust, style=rust,
firstline=293, lastline=309]{../../src/dbscan/para_model.rs}
\newpage
\lstinputlisting[caption={Параллельный алгоритм DBSCAN, получение ячеек}, label={lst:makeCell}, language=Rust, style=rust,
firstline=139, lastline=211]{../../src/dbscan/para_model.rs}
\newpage
\lstinputlisting[caption={Параллельный алгоритм DBSCAN, получение coreFlags}, label={lst:markCore}, language=Rust, style=rust,
firstline=311, lastline=353]{../../src/dbscan/para_model.rs}
\newpage
\lstinputlisting[caption={Параллельный алгоритм DBSCAN, получение кластеров}, label={lst:clusterCore}, language=Rust, style=rust,
firstline=369, lastline=442]{../../src/dbscan/para_model.rs}
\newpage
\lstinputlisting[caption={Параллельный алгоритм DBSCAN, расширение кластеров}, label={lst:clusterBorder}, language=Rust, style=rust,
firstline=446, lastline=480]{../../src/dbscan/para_model.rs}
\newpage
\section{Тестовые данные}

Одним из недостатков алгоритма DBSCAN является его неодназначность, т.е. нет гарантии в однозначности результата. В связи с этим провести сравнение с неким эталонным результатом не представляется возможным

\section*{Вывод}

В данном разделе был продемонстрирован исходный код алгоритма DBSACN, последовательного и параллельного.

