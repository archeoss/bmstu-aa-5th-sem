\chapter{Аналитическая часть}

\section{Алгоритм DBSCAN}
Идея, положенная в основу алгоритма, заключается в том, что внутри каждого кластера наблюдается типичная плотность точек (объектов), которая заметно выше, чем плотность снаружи кластера, а также плотность в областях с шумом ниже плотности любого из кластеров. 

Ещё точнее, что для каждой точки кластера её соседство заданного радиуса должно содержать не менее некоторого числа точек, это число точек задаётся пороговым значением. 
Перед изложением алгоритма дадим необходимые определения.\\

\textbf{Определение 1.} Eps-соседство точки \(p\), обозначаемое как \(N_{\varepsilon}(p)\), определяется как множество документов, находящихся от точки \(p\) на расстояния не более \(\varepsilon\): 
\(N_{\varepsilon}(p) = \{ q \in D | dist(p, q) \le \varepsilon \}\). Поиска точек, чьё \(N_{\varepsilon}(p)\) содержит хотя бы минимальное число точек (MinPt) не достаточно, так как точки бывают двух видов: 
ядровые и граничные.\\
\indent\textbf{Определение 2.} Точка \(p\) непосредственно плотно-достижима из точки \(q\) (при заданных \(\varepsilon\) и MinPt), если \( p \in N_{\varepsilon}(p) \) и \(|N_{\varepsilon}(p)| \ge MinPt \) (рис. \ref{img:dens}) 
\includeimage
{dens} % Имя файла без расширения (файл должен быть расположен в директории inc/img/)
{f} % Обтекание (без обтекания)
{h} % Положение рисунка (см. figure из пакета float)
{0.75\textwidth} % Ширина рисунка
{Пример точек, находящихся в отношении непосредственно плотной достижимости} % Подпись рисунка
% \clearpage

\textbf{Определение 3.} Точка \(p\) плотно-достижима из точки \(q\) (при заданных \(\varepsilon\) и MinPt), если существует последовательность точек \(q = p_1, p_2, ... , p_n = p: p_{i+1}\) непосредственно плотно-достижимы из \(p_i\). 
Это отношение транзитивно, но не симметрично в общем случае, однако симметрично для двух ядровых точек.\\ 
\clearpage
\indent\textbf{Определение 4.} Точка \(p\) плотно-связана с точкой \(q\) (при заданных \(\varepsilon\) и MinPt), если существует точка \(o: p\, \text{и}\, q\) плотно-достижимы из \(o\) (при заданных \(\varepsilon\) и MinPt), (рис. \ref{img:dens_connect})
\includeimage
{dens_connect} % Имя файла без расширения (файл должен быть расположен в директории inc/img/)
{f} % Обтекание (без обтекания)
{h} % Положение рисунка (см. figure из пакета float)
{0.75\textwidth} % Ширина рисунка
{Пример точек, находящихся в отношении плотной связанности.} % Подпись рисунка
% \clearpage

Теперь мы готовы дать определения кластеру и шуму.\\ 
\indent\textbf{Определение 5.} Кластер \(\mathcal{C}_j\) (при заданных \(\varepsilon\) и MinPt) – это не пустое подмножество документов, удовлетворяющее следующих условиям: 
\begin{enumerate}
    \item  \(\forall p,q: \text{если } p \in \mathcal{C}_j \,\text{и}\, q \text{ плотно-достижима из } p \text{(при заданных \(\varepsilon\) и MinPt)},\\ \text{то }q \in \mathcal{C}_j\),
    \item  \(\forall p,q \in \mathcal{C}_j : p \,\text{плотно связана с}\, q \text{ (при заданных \(\varepsilon\) и MinPt)}\);\\
\end{enumerate} 

\indent Итак, кластер – это множество плотно-связанных точек. В каждом кластере 
содержится хотя бы MinPt документов. \\
\indent\textit{Шум} – это подмножество документов, которые не принадлежат ни одному 
кластеру: \{\( p \in D |  \forall j: p \notin \mathcal{C}_j , j = \overline{ 1, |\mathcal{C}|}\)\}\\
\indent Алгоритм DBSCAN для заданных значений параметров \(\varepsilon\) и MinPt исследует кластер следующим образом: сначала выбирает случайную точку, являющуюся ядровой, в качестве затравки, затем помещает в кластер саму затравку и все точки, плотно-достижимые из неё. 
\clearpage
\textbf{Алгоритм в общем виде.}

\begin{table}[h!]
        \begin{threeparttable}
            \captionsetup{justification=raggedleft,singlelinecheck=off}
            \begin{tabular}{l}
                \hline
                DBSCAN \\
                \hline
                Вход: множество точек документов \(\mathcal{D}\), \(\varepsilon\) и MinPt. \\
                Выход: множество кластеров \(\mathcal{C} = \{C_j\}\).\\
                Шаг 1. Установить всем элементам множества \(\mathcal{D}\) флаг «не посещён». \\
                Присвоить текущему кластеру \(\mathcal{C}_j\) нулевой номер, \(j := 0\). \\
                Множество шумовых документов Noise := \(\varnothing\).\\
                Шаг 2. Для каждого \(d_i \in \mathcal{D}\) такого, что флаг(\(d_i\)) = «не посещён», выполнить: \\
                Шаг 3. \tab флаг(\(d_i\)) := «посещён»; \\
                Шаг 4. \tab \(N_i := N_{\varepsilon}(d_i) = \{q \in \mathcal{D} | dist(d_i,q) \le \varepsilon\}\) \\
                Шаг 5. \tab Если|\(N_i\)| < MinPt, то  \\
                \tab\tab\tab\tab Noise := Noise + \(\{d_i\}\) \\
                \tab\tab иначе \\
                \tab\tab\tab\tab номер следующего кластера j := j + 1;  \\
                \tab\tab\tab\tab EXPANDCLUSTER(\(d_i, N_i, \mathcal{C}, \varepsilon, MinPt\)); \\
                \hline
    %            \vspace{2mm}
            \end{tabular}
        \end{threeparttable}
\end{table}

\begin{table}[h!]
        \begin{threeparttable}
            \captionsetup{justification=raggedleft,singlelinecheck=off}
            \begin{tabular}{l}
                \hline
                EXPANDCLUSTER \\
                \hline
                Вход: текущая точка \(d_i\), его \(\varepsilon\)-соседство \(N_i\), текущий кластер \(C_j\) и \(\varepsilon\), MinPt.\\
                Выход: кластер \(C_j\)\\
                Шаг 1. \(C_j := C_j + \{d_i\}\); \\
                Шаг 2. Для всех документов \(d_k \in N_i\): \\
                Шаг 3. \tab Если флаг(\(d_k\)) = «не посещён», то \\
                Шаг 4. \tab\tab флаг(\(d_k\)) := «посещён»; \\
                Шаг 5. \tab\tab \(N_{ik} := N_{\varepsilon}(d_k)\); \\
                Шаг 6. \tab\tab \(Если |N_{ik}| \ge MinPt, \text{ то } N_i := N_i + N_{ik}\); \\
                Шаг 7. \tab Если \(\nexists p: d_k \in C_p, p = \overline{1, |\mathcal{C}|}, \text{ то } C_j := C_j = \{d_k\}\); \\
                \hline
            \end{tabular}
        \end{threeparttable}
\end{table}
\clearpage
\section{Параллельный алгоритм DBSCAN}
Параллельный алгоритм DBSCAN заключается в том, что точки помещаются в непересекающиеся \(d\)-мерные клетки с длиной стороны \(\varepsilon / \sqrt{d}\) по их координатам (рис. \ref{img:cells}(b)).
Ячейки обладают тем свойством, что все точки внутри клетки находятся на расстоянии \(\varepsilon\) друг от друга и будут принадлежать к одному и тому же кластеру.
Затем (рис. \ref{img:cells}(c)) мы отмечаем основные точки, после чего мы генерируем кластеры для основных точек следующим образом:\\
\indent Мы создаем граф содержащий по одной вершине на ядровую ячейку (ячейка, содержащая не менее одной ядровой точки), и соединить две вершины, если ближайшая пара ядровых точек из двух ячеек находится на расстоянии \(\varepsilon\).
Далее будем сслыаться на него как как на клеточный граф.
Сам этот шаг проиллюстрирован на рисунке \ref{img:cells}(d). 
Затем находим компоненты связности графа ячеек для назначения идентификаторов кластеров точкам в ядровых ячейках и назначаем идентификаторы кластеров для граничных точек. 
В конце возращаем список кластеров. 


\includeimage
{cells} % Имя файла без расширения (файл должен быть расположен в директории inc/img/)
{f} % Обтекание (без обтекания)
{h} % Положение рисунка (см. figure из пакета float)
{0.75\textwidth} % Ширина рисунка
{Параллельный алгоритм DBSCAN} % Подпись рисунка

\clearpage

\textbf{Алгоритм в общем виде.}
\begin{table}[h!]
        \begin{threeparttable}
            \captionsetup{justification=raggedleft,singlelinecheck=off}
            \begin{tabular}{p{16cm}}
                \hline
                DBSCAN \\
                \hline
                Вход: множество индексированных точек \(\mathcal{P}\), \(\varepsilon\) и MinPt. \\
                Выход: множество кластеров \(\mathcal{C} = \{C_j\}\).\\
                Шаг 1. Разбить множество \(\mathcal{P}\) точек на множество ячеек \(\mathcal{G}\)\\
                Шаг 2. coreFlags := MarkCore(\(\mathcal{G}, \mathcal{P}, \varepsilon, MinPt\))\\
                Шаг 3. \(\mathcal{C}\) := ClusterCore(\(\mathcal{G}, \mathcal{P}, coreFlags, \varepsilon, MinPt\))\\
                Шаг 4. ClusterBorder(\(\mathcal{G}, \mathcal{P}, coreFlags, \mathcal{C}, \varepsilon, MinPt\))\\
                Шаг 5. Вернуть \(\mathcal{C}\) \\
                \hline
    %            \vspace{2mm}
            \end{tabular}
        \end{threeparttable}
\end{table}

\begin{table}[h!]
        \begin{threeparttable}
            \captionsetup{justification=raggedleft,singlelinecheck=off}
            \begin{tabular}{p{16cm}}
                \hline
                MarkCore \\
                \hline
                Вход: множество индексированных точек \(\mathcal{P}\), множество ячеек \(\mathcal{G}\) \(\varepsilon\) и MinPt. \\
                Выход: список ядровых точек \(coreFlags\).\\
                1. \(coreFlags := \{0, ..., 0\}\) \tab\tab\tab \(\backslash\backslash\) Длина равна размеру \(\mathcal{P}\) \\
                2. Параллельно \(\forall g \in \mathcal{G}\) \\
                3. \tab Если \(len(g) \ge minPt\)\\
                4. \tab\tab Для \(\forall p \in g\)\\
                5. \tab\tab\tab \(coreFlags[p] := 1\)\\
                6. \tab\tab иначе \\
                7. \tab\tab\tab для \(\forall p \in g\)\\
                8. \tab\tab\tab\tab \(count:= len(g) \)\\
                9. \tab\tab\tab\tab для \(\forall h \in g.NeighborCells(\varepsilon)\) \\
                10. \tab\tab\tab\tab\tab \(count := count + RangeCount(p, \varepsilon, h) \)\\
                11. \tab\tab\tab\tab Если \(count \ge minPt\)\\
                12. \tab\tab\tab\tab\tab \(coreFlags[p] := 1\)\\
                13. Вернуть \(coreFlags\)\\
                \hline
    %            \vspace{2mm}
            \end{tabular}
        \end{threeparttable}
\end{table}

\begin{table}[h!]
        \begin{threeparttable}
            \captionsetup{justification=raggedleft,singlelinecheck=off}
            \begin{tabular}{p{16cm}}
                \hline
                ClusterCore \\
                \hline
                Вход: множество индексированных точек \(\mathcal{P}\), множество ячеек \(\mathcal{G}\), \\
                \tab\tab\tab\tab\tab\tab флаги ядровых ячеек \(coreFlags\), \(\varepsilon\) и MinPt. \\
                Выход: Кластеры \(\mathcal{C}\).\\
                1. uf := \(UnionFind()\)\tab\tab \(\backslash\backslash\)Инициализация Union-find структуры\\
                2. Параллельно \(\forall g \in \mathcal{G}\) \\
                3. \tab Для \(\forall h \in g.NeighborCells()\) \\
                4. \tab\tab Если \(g > h\quad \&\&\quad uf.Find(g) \neq uf.Find(h)\) \\
                5. \tab\tab\tab Если \(Connected(g, h)\)\\
                6. \tab\tab\tab\tab \(uf.Link(g, h)\)\\
                7. \tab\tab\tab для \(\forall p \in g\)\\
                8. \(\mathcal{C} := \{-1,...,-1\}\)\tab\tab\tab Длина множества \(\mathcal{P}\) \\
                9. Параллельно \(\forall g \in \mathcal{G}\) \\
                10. \tab \(\forall p \in g : coreFlags[p] = 1\)\\
                11. \tab\tab \(\mathcal{C}[p] := uf.Find(g)\)\\
                12. Вернуть \(\mathcal{C}\)\\
                \hline
    %            \vspace{2mm}
            \end{tabular}
        \end{threeparttable}
\end{table}

\begin{table}[h!]
        \begin{threeparttable}
            \captionsetup{justification=raggedleft,singlelinecheck=off}
            \begin{tabular}{p{16cm}}
                \hline
                ClusterBorder \\
                \hline
                Вход: множество индексированных точек \(\mathcal{P}\), множество ячеек \(\mathcal{G}\), \\
                \tab\tab\tab кластеры \(\mathcal{C}\), флаги ядровых ячеек \(coreFlags\), \(\varepsilon\) и MinPt. \\
                Выход: Обновленные кластеры \(\mathcal{C}\)\\ 
                1. Параллельно \(\forall g \in \mathcal{G} : len(g) < minPt\) \\
                2. \tab Для \(\forall p \in cell\, g : coreFlags[p] = 0\) \\
                3. \tab\tab Для \(\forall h \in g \cup g.NeighborCells(\varepsilon)\)\\ 
                4. \tab\tab\tab Для \(\forall q \in cell\, h : coreFlags[q] = 1\)\\
                5. \tab\tab\tab\tab Если \(dist(p,q) \le \varepsilon\)\\
                6. \tab\tab\tab\tab\tab \(\mathcal{C}[p] := \mathcal{C}[p] \cup \mathcal{C}[q]\)\\
                7. Вернуть \(\mathcal{C}\)\\
                \hline
    %            \vspace{2mm}
            \end{tabular}
        \end{threeparttable}
\end{table}
\FloatBarrier


\section*{Вывод}
В данном разделе были рассмотрены алгоритмы DBSCAN и его распараллеленного аналога.
\clearpage
