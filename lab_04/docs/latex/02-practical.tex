\chapter{Конструкторская часть}

\section{Разработка алгоритмов}

На рисунках \ref{img:baseSerial}, \ref{img:expandclusters} представлена схема линейного алгоритма DBSCAN. 

На рисунках \ref{img:baseParallel}, \ref{img:markCore}, \ref{img:markCoreA}, \ref{img:clusterCore}, \ref{img:clusterCoreB}, \ref{img:clusterCoreC} представлена схема параллельного алгоритма DBSCAN.

\includeimage
{baseSerial} % Имя файла без расширения (файл должен быть расположен в директории inc/img/)
{f} % Обтекание (без обтекания)
{h} % Положение рисунка (см. figure из пакета float)
{0.65\textwidth} % Ширина рисунка
{Линейный алгоритм DBSCAN} % Подпись рисунка
\clearpage

\includeimage
{expandclusters} % Имя файла без расширения (файл должен быть расположен в директории inc/img/)
{f} % Обтекание (без обтекания)
{h} % Положение рисунка (см. figure из пакета float)
{0.65\textwidth} % Ширина рисунка
{Функция EXPANDCLUSTERS линейного алгоритма DBSCAN} % Подпись рисунка
\clearpage

\includeimage
{baseParallel} % Имя файла без расширения (файл должен быть расположен в директории inc/img/)
{f} % Обтекание (без обтекания)
{h} % Положение рисунка (см. figure из пакета float)
{0.65\textwidth} % Ширина рисунка
{Параллельный алгоритм DBSCAN} % Подпись рисунка
\clearpage

\includeimage
{markCore} % Имя файла без расширения (файл должен быть расположен в директории inc/img/)
{f} % Обтекание (без обтекания)
{h} % Положение рисунка (см. figure из пакета float)
{0.65\textwidth} % Ширина рисунка
{Функция markCore параллельного алгоритма DBSCAN, часть 1} % Подпись рисунка
\clearpage

\includeimage
{markCoreA} % Имя файла без расширения (файл должен быть расположен в директории inc/img/)
{f} % Обтекание (без обтекания)
{h} % Положение рисунка (см. figure из пакета float)
{0.55\textwidth} % Ширина рисунка
{Функция markCore параллельного алгоритма DBSCAN, часть 2} % Подпись рисунка
\clearpage

\includeimage
{clusterCore} % Имя файла без расширения (файл должен быть расположен в директории inc/img/)
{f} % Обтекание (без обтекания)
{h} % Положение рисунка (см. figure из пакета float)
{0.65\textwidth} % Ширина рисунка
{Функция clusterCore параллельного алгоритма DBSCAN, часть 1} % Подпись рисунка
\clearpage

\includeimage
{clusterCoreB} % Имя файла без расширения (файл должен быть расположен в директории inc/img/)
{f} % Обтекание (без обтекания)
{h} % Положение рисунка (см. figure из пакета float)
{0.65\textwidth} % Ширина рисунка
{Функция clusterCore параллельного алгоритма DBSCAN, часть 2} % Подпись рисунка
\clearpage

\includeimage
{clusterCoreC} % Имя файла без расширения (файл должен быть расположен в директории inc/img/)
{f} % Обтекание (без обтекания)
{h} % Положение рисунка (см. figure из пакета float)
{0.65\textwidth} % Ширина рисунка
{Функция clusterCore параллельного алгоритма DBSCAN, часть 3} % Подпись рисунка
\clearpage

\section*{Вывод}
На основе теоретических данных, полученных из аналитического раздела, были построены схемы алгоритмов DBSCAN, последовательного и параллельного.

