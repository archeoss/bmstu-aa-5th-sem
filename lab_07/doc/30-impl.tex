\chapter{Технологическая часть}

В данном разделе приведены средства программной реализации и листинг кода.

\section{Требования к ПО}

К программе предъявляется ряд требований:
\begin{itemize}
    \item на вход подается запрос на естественном русском языке;
    \item все запросы должны быть связаны с участниками стрелковых соревнований и их меткостью;
    \item на выход программа должна выдать ответ, основанный на данных словаря.
\end{itemize}

\section{Листинг кода}

В листингах \ref{lst:maps} приведена реализация словарей.

\begin{lstinputlisting}[
	caption={Реализация словарей.},
	label={lst:maps},
	style={rust}
]{../src/lib/maps.rs}
\end{lstinputlisting}

\begin{lstinputlisting}[
	caption={Реализация термов.},
	label={lst:terms},
	style={rust}
]{../src/lib/terms.rs}
\end{lstinputlisting}


\section{Тестирование функций.}

В таблице \ref{tab:tests} представлены данные для тестирования. Все тесты пройдены успешно.

\begin{table}[h!]
	\begin{center}
		\begin{tabular}{|p{8cm}  | p{8cm}  |}
            \hline
            Запрос & Ожидание \\
            \hline
            Кто из стрелков очень меткий? & "VeryAccurate на промежутке [0, 2.875) \\
            \hline
            Выбрать всех стрелков, которые попадают между категориями Меткий и Не очень меткий?  &  Accurate на промежутке [2.875, 6), NotSoAccurate на промежутке [6, 7)\\
            \hline
            Меткие шутники & Введена неверная строка, повторите запрос. \\
            \hline
		\end{tabular}
	\end{center}
	\caption{\label{tab:tests} Тестирование функций.}
\end{table}
\FloatBarrier
\section*{Вывод}

Была разработана и протестирована реализация словарей.
