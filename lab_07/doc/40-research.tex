\chapter{Исследовательская часть}

Ниже приведены технические характеристики устройства, на котором было проведено тестирование ПО.

\begin{itemize}
    \item[$-$] Операционная система: Arch Linux~\cite{arch} 64-bit.
    \item[$-$] Оперативная память: 16 Гбайт, DDR4.
    \item[$-$] Процессор: 11th Gen Intel\textsuperscript{\tiny\textregistered} Core\textsuperscript{\tiny\texttrademark} i5-11320H @ 3.20 ГГц~\cite{i5}. 
\end{itemize}

\section{Формализация объекта и его признака}\label{formal}

Согласно согласованному варианту, формализуем объект <<Стрелок>> следующим образом: 
определим набор данных и признак объекта, на основании которого составим набор терм.

Набор данных для объекта:
\begin{itemize}
	\item[$-$] Имя стрелка --- строка;
	\item[$-$] Меткость в MOA --- число с плавающей точкой;
\end{itemize}

Согласно варианту, числовым признаком, по которому будет производиться поиск объектов, меткость. 

Определим следующие термы, соответствующие признаку <<меткость>>:
\begin{enumerate}[ 1{)}] 
    \item ОМ --- очень меткий;
    \item М --- меткий;
    \item НОМ --- не очень меткий; 
    \item НМ --- неметкий; 
    \item ОНМ --- очень неметкий. 
\end{enumerate}

\clearpage

\section{Анкетирование}

Было проведено анкетирование следующих респондентов:
\begin{enumerate}[ 1{)}]
    \item Назиров Илхом, группа ИУ7-55Б --- Респондент 1;
	\item Иванов Павел, группа ИУ7-55Б--- Респондент 2;
	\item Щербина Михалил, группа ИУ7-55Б --- Респондент 3;
	\item Вязовцев Максим, группа ИУ7-55Б --- Респондент 4;
	\item Мансуров Владислав, группа ИУ7-56Б --- Респондент 5;
	\item Голикова Софья, группа ИУ7-55Б --- Респондент 6;
	\item Баринов Никита, группа ИУ7-51Б --- Респондент 7;
	\item Загайнов Никита, группа ИУ7-52Б --- Респондент 8;
	\item Ильин Дмитрий, группа ИУ7-55Б --- Респондент 9.
\end{enumerate}

Респонденты отвечали на вопросы, сформулированные в пункте \ref{sec:anketa}
Статистика результатов анкетирования перечисленных респондентов продемонстрированы в таблицах~\ref{tbl:result_application_1} -- \ref{tbl:result_application_2}. 
В данных таблицах термы соответствуют обозначенным в \ref{formal} сокращенным термам.

\clearpage
\begin{table}[ht]
	\small
	\begin{center}
		\begin{threeparttable}
		\caption{Результаты анкетирования (Часть 1)}
		\label{tbl:result_application_1}
		\begin{tabular}{|p{1.5cm}|p{2cm}|p{2cm}|p{2cm}|p{2cm}|p{2cm}|p{2cm}|}
			\hline
			\multirow{\bfseries Терма} & \multicolumn{6}{c|}{\bfseries Категория} \\ \cline{2-7}
			& \bfseries Проф. снайпер & \bfseries Спец. SWAT & \bfseries Опыт. Стрелок & \bfseries Поли-цейский & \bfseries Стрелок-любитель & \bfseries Срочник \\
			\hline
			ОМ & \text{9} & \text{8} & \text{5} & \text{1} & \text{0} & \text{0} \\
			\hline
			М & \text{0} & \text{1} & \text{4} & \text{5} & \text{5} & \text{3} \\
			\hline
			НОМ & \text{0} & \text{0} & \text{0} & \text{3} & \text{4} & \text{5} \\
			\hline
			НМ & \text{0} & \text{0} & \text{0} & \text{0} & \text{0} & \text{1} \\
			\hline
			ОНМ & \text{0} & \text{0} & \text{0} & \text{0} & \text{0} & \text{0} \\
			\hline
			\end{tabular}
		\end{threeparttable}
	\end{center}
\end{table}

\begin{table}[ht]
	\small
	\begin{center}
		\begin{threeparttable}
		\caption{Результаты анкетирования (Часть 2)}
		\label{tbl:result_application_2}
		\begin{tabular}{|p{1.5cm}|p{3cm}|p{3cm}|p{3cm}|p{3cm}|}
			\hline
		\multirow{\bfseries Терма} & \multicolumn{4}{c|}{\bfseries Категория} \\ \cline{2-5}
			 & \bfseries Гук & \bfseries Имперский штурмовик  & \bfseries Африканский повстанец  & \bfseries Китайский фермер \\
			\hline
			ОМ & \text{0} & \text{0} & \text{0} & \text{0} \\% & \text{0} & \text{0} \\
			\hline
			М & \text{0} & \text{0} & \text{0} & \text{0} \\% & \text{5} & \text{3} \\
			\hline
			НОМ & \text{3} & \text{0} & \text{0} & \text{0} \\% & \text{4} & \text{5} \\
			\hline
			НМ & \text{5} & \text{6} & \text{2} & \text{0} \\% & \text{0} & \text{1} \\
			\hline
			ОНМ & \text{1} & \text{3} & \text{7} & \text{9} \\% & \text{0} & \text{0} \\
			\hline
			\end{tabular}
		\end{threeparttable}
	\end{center}
\end{table}
\newpage
\begin{figure}
\begin{tikzpicture}[scale=1.2]\label{}
  \begin{axis}
        [
        ,width=12cm
        ,xlabel=Категория стрелков
        ,ylabel=$\mu$
        ,xtick=data,
       % ,xtick={0,1,2}
       % ,ytick={0,1,2,3,4,5,6,7,8,9}
       ,xticklabels={A, B, C, D, E, F, G, H, I, G}
        ]
        \addplot+[sharp plot, legend entry=ОМ ] coordinates
        {(0, 1.0) (1, 0.888) (2, 0.444) (3, 0.111) (4, 0) (5, 0) (6, 0) (7, 0) (8, 0) (9, 0)};
        \addplot+[sharp plot, legend entry=М  ] coordinates
        {(0, 0.0) (1, 0.111) (2, 0.555) (3, 0.666) (4, 0.555) (5, 0.444) (6, 0) (7, 0) (8, 0) (9, 0)};
        \addplot+[sharp plot, legend entry=НОМ ] coordinates
        {(0, 0.0) (1, 0.0) (2, 0.0) (3, 0.222) (4, 0.444) (5, 0.444) (6, 0.444) (7, 0) (8, 0) (9, 0)};
        \addplot+[sharp plot, legend entry=Н ] coordinates
        {(0, 0.0) (1, 0.0) (2, 0.0) (3, 0) (4, 0) (5, 0.111) (6, 0.444) (7, 0.666) (8, 0.222) (9, 0)};
        \addplot+[sharp plot, green, legend entry=ОНМ ] coordinates
        {(0, 0.0) (1, 0.0) (2, 0.0) (3, 0) (4, 0) (5, 0) (6, 0.111) (7, 0.333) (8, 0.777) (9, 1)};
    \end{axis}
\end{tikzpicture}
\caption{Функция принадлежности термам числовых значений признака}
\end{figure}
Обозначения:
\begin{enumerate}[ 1{)}] 
    \item A --- Проф. снайпер
    \item B --- Спец. SWAT 
    \item C --- Опыт. Стрелок 
    \item D --- Полицейский 
    \item E --- Стрелок-любитель 
    \item F --- Срочник
    \item G --- Гук
    \item H --- Имперский штурмовик 
    \item I --- Африканский повстанец  
    \item G --- Китайский фермер
\end{enumerate}

На рисунке 5 параметр µ означает частоту встречи (степень принадлеж-ности) терма для каждого напитка. Чем больше µ, тем чаще эксперты называли
терм в отношении конкретного признака объекта. Таким образом, на рисунке 5
представлены нечеткие функции принадлежности объектов термам ti.
В соответствии с полученным графиком будем считать меткость следующим образом:
\begin{enumerate}[ 1{)}]
	\item очень меткий, если значение точности лежит в промежутке $[1.0; 2.875)$ MOA;
	\item меткий, если значение точности лежит в промежутке $[2.875; 6)$ MOA;
	\item не очень меткий, если значение точности лежит в промежутке $[6; 7)$ MOA;
	\item неметкий, если значение точности лежит в промежутке $[7; 8.375)$ MOA;
	\item очень неметкий, если значение их точности лежит в промежутке $[8.375; +\infty)$ MOA.
\end{enumerate}

\section{Пример выполнения}

На рисунках \ref{img:example_simple} -- \ref{img:example_somebody} представлены примеры работы выполнения программы.
\img{100mm}{example_simple}{Простой поиск по термам}% \newpage
\img{100mm}{example_between}{Поиск между термами}% \newpage
\img{80mm}{example_somebody}{Поиск по стрелкам} %\newpage
\clearpage

\section*{Вывод}
В данном разделе были приведены результаты опроса респондентов. 
На основе полученных данных сделан вывод о соотнесении признака объекта <<стрелок>> с термами, характеризующие меткость: <<очень меткий>>, <<меткий>>, <<не очень меткий>>, <<неметкий>>, <<очень неметкий>>.
