\chapter{Аналитическая часть}

Словарь (или ``\textit{ассоциативный массив}'')\cite{dict} - абстрактный тип данных (интерфейс к хранилищу данных), позволяющий хранить пары вида «(ключ, значение)» и поддерживающий операции добавления пары, а также поиска и удаления пары по ключу:
\begin{itemize}
    \item \texttt{ВСТАВКА(ключ, значение)};
    \item \texttt{ПОИСК(ключ)};
    \item \texttt{УДАЛЕНИЕ(ключ)}.
\end{itemize}

В паре \texttt{(k, v)} значение \texttt{v} называется значением, ассоциированным с ключом \texttt{k}. Где \texttt{k} — это ключ, a \texttt{v} — значение. Семантика и названия вышеупомянутых операций в разных реализациях ассоциативного массива могут отличаться.

Операция \texttt{ПОИСК(ключ)} возвращает значение, ассоциированное с заданным ключом, или некоторый специальный объект \texttt{НЕ\_НАЙДЕНО}, означающий, что значения, ассоциированного с заданным ключом, нет. Две другие операции ничего не возвращают (за исключением, возможно, информации о том, успешно ли была выполнена данная операция).

Ассоциативный массив с точки зрения интерфейса удобно рассматривать как обычный массив, в котором в качестве индексов можно использовать не только целые числа, но и значения других типов — например, строки.

\section{Алгоритм полного перебора}

Алгоритмом полного перебора \cite{brute} называют метод решения задачи, при котором по очереди рассматриваются все возможные варианты. В нашем случае мы последовательно будем перебирать ключи словаря до тех пор, пока не найдём нужный. Трудоёмкость алгоритма зависит от того, присутствует ли искомый ключ в словаре, и, если присутствует - насколько он далеко от начала массива ключей.

Пусть алгоритм нашёл элемент на первом сравнении (лучший случай), тогда будет затрачено $k_0 + k_1$ операций, на втором - $k_0 + 2 \cdot k_1$, на последнем (худший случай) - $k_0 + N \cdot k_1$. Если ключа нет в массиве ключей, то мы сможем понять это, только перебрав все ключи, таким образом трудоёмкость такого случая равно трудоёмкости случая с ключом на последней позиции. Средняя трудоёмкость может быть рассчитана как математическое ожидание по формуле (\ref{for:brute}), где $\Omega$ -- множество всех возможных случаев.

\begin{equation}
    \label{for:brute}
    \begin{aligned}
        \sum\limits_{i \in \Omega} p_i \cdot f_i = & (k_0 + k_1) \cdot \frac{1}{N + 1} + (k_0 + 2 \cdot k_1) \cdot \frac{1}{N+1} +\\
        & + (k_0 + 3 \cdot k_1) \cdot \frac{1}{N + 1} + (k_0 + Nk_1)\frac{1}{N + 1} + (k_0 + N \cdot k_1) \cdot \frac{1}{N + 1} =\\
        & = k_0\frac{N+1}{N+1}+k_1+\frac{1 + 2 + \cdots + N + N}{N + 1} = \\
        & = k_0 + k_1 \cdot \left(\frac{N}{N + 1} + \frac{N}{2}\right) = k_0 + k_1 \cdot \left(1 + \frac{N}{2} - \frac{1}{N + 1}\right)
    \end{aligned}
\end{equation}

\section{Формализация объекта}
Объектами, с которыми связано решение задачи, являются участники стрелк\'{о}вых соревнований.
Одним из важнейших критериев для всех стрелков является их меткость.
В данном конкретном случае <<меткость>> измеряется в так называемых MOA (Minute of Angle или Угловая минута) --- отклонение фактического попадания от центра мишени. 
Одна угловая минута примерно равна 1 дюйму на расстоянии 100 ярдов, 2 дюймам на 200 ярдов и т.д.
Так например, на текущий момент стандартом, или точнее необходимым минимумом, в различных армейских частях является попадние в мишень в рамках 6 MOA на протяжении 5 выстрелов подряд.
В таблице \ref{tab:moa} представлена средняя меткость по выбранной группе лиц.
\begin{table}[h!]
	\caption{\label{tab:moa} Меткость стрелков.}
	\begin{center}
		\begin{tabular}{| p{10cm} | l |}
            \hline
            Категория & Меткость \\
            \hline
            Профессиональный снайпер & 1.0 MOA \\
            \hline
            Спецназ SWAT & 2.0 MOA \\
            \hline
            Опытный стрелок & 3.0 MOA \\
            \hline
            Полицейский & 4.0 MOA \\
            \hline
            Стрелок-любитель & 5.0 MOA \\
            \hline
            Солдат-срочник & 6.0 MOA \\
            \hline
            Гук & 7.0 MOA \\
            \hline
            Имперский штурмовик & 8.0 MOA \\
            \hline
            Африканский повстанец & 9.0 MOA \\
            \hline
            Китайский фермер & 10.0 MOA \\
            \hline
		\end{tabular}
	\end{center}
\end{table}

\section*{Вывод}
В данном разделе был рассмотрен абстрактный тип данных словарь, возможные реализации поиска в нём, формализован объект <<участник стрелковых соревнований>> и его признак --- меткость, определяемая в MOA.

