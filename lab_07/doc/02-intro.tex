\chapter*{Введение}
\addcontentsline{toc}{chapter}{Введение}
В процессе развития технологий объемы обрабатываемых данных постоянно растут. 
Это привело к увеличению времени выполнения операций над наборами данных \cite{def1}. 
Таким образом, более внимательно стали рассматриваться алгоритмы для работы с коллекциями, а также появилась необходимость в
создании новых алгоритмов, которые решают поставленную задачу более эффективно по затрачиваемым ресурсам. 
В том числе это касается и словарей, в которых одной из основных операций является операция поиска.
Цель данной лабораторной работы — описание и реализация алгоритма поиска по словарю при ограничении на значение признака, заданном при помощи лингвистической переменной.
Для достижения данной цели необходимо решить следующие задачи:
\begin{enumerate}[ 1{)}] 
	\item формализовать объект и его признак;
	\item составить анкету для заполнения респондентом;
	\item провести анкетирование респондентов;
	\item построить функцию принадлежности термам числовых значений признака, описываемого лингвистической переменной, на основе статистической обработки мнений респондентов, выступающих в роли экспертов;
	\item описать 3–5 типовых вопросов на русском языке, имеющих целью запрос на поиск в словаре;
	\item описать алгоритм поиска в словаре объектов, удовлетворяющих ограничению, заданному в вопросе на ограниченном естественном языке;
	\item описать структуру данных словаря, хранящего наименование объектов согласно варианту и числовое значение признака объекта;
	\item реализовать описанный алгоритм поиска в словаре;
	\item привести примеры запросов пользователя и сформированной реализацией алгоритма поиска выборки объектов из словаря, используя составленные респондентами вопросы;
	\item дать заключение о применимости предложенного алгоритма и его ограничениях;
	\item описать и обосновать полученные результаты в виде отчета о выполненной лабораторной работе, выполненном как расчетно-пояснительная записка к работе.
\end{enumerate}
