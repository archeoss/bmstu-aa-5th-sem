\chapter{Конструкторская часть}
В данном разделе представлена схема алгоритма поиска по словарю, анкета для заполнения респондентами,
сформулированы типовые запросы на русском языке для поиска по словарю.
\section{Анкета для респондентов}\label{sec:anketa}
Для ответа на вопросы, уточняющие точность выбранной категории лиц, разработана следующая анкета для респондентов.
\begin{enumerate}[ 1{)}]
  \item Оцените меткость следующей категории лиц: Профессиональный снайпер. Варианты ответа: Очень меткий, Меткий, Не очень меткий, Не меткий, Очень не меткий;
  \item Оцените меткость следующей категории лиц: Спецназ SWAT. Варианты ответа: Очень меткий, Меткий, Не очень меткий, Не меткий, Очень не меткий;
  \item Оцените меткость следующей категории лиц: Опытный стрелок. Варианты ответа: Очень меткий, Меткий, Не очень меткий, Не меткий, Очень не меткий;
  \item Оцените меткость следующей категории лиц: Полицейский. Варианты ответа: Очень меткий, Меткий, Не очень меткий, Не меткий, Очень не меткий;
  \item Оцените меткость следующей категории лиц: Стрелок-любитель. Варианты ответа: Очень меткий, Меткий, Не очень меткий, Не меткий, Очень не меткий;
  \item Оцените меткость следующей категории лиц: Солдат-срочник. Варианты ответа: Очень меткий, Меткий, Не очень меткий, Не меткий, Очень не меткий;
  \item Оцените меткость следующей категории лиц: Гук. Варианты ответа: Очень меткий, Меткий, Не очень меткий, Не меткий, Очень не меткий;
  \item Оцените меткость следующей категории лиц: Имперский штурмовик. Варианты ответа: Очень меткий, Меткий, Не очень меткий, Не меткий, Очень не меткий;
  \item Оцените меткость следующей категории лиц: Африканский повстанец. Варианты ответа: Очень меткий, Меткий, Не очень меткий, Не меткий, Очень не меткий;
  \item Оцените меткость следующей категории лиц: Китайский фермер. Варианты ответа: Очень меткий, Меткий, Не очень меткий, Не меткий, Очень не меткий.
\end{enumerate}

\section{Типовые запросы}
Примеры запросов, на которые должна отвечать система, в соответствии
с собранными данными, следующие.
\begin{enumerate}[ 1{)}]
  \item Кто из стрелков очень меткий?
  \item Кто из стрелков стреляет лучше, чем неметкие?
  \item Выбрать всех стрелков, которые попадают между категориями Меткий и Не очень меткий?
  \item Выбрать всех стрелков, которые стреляют как \{ Список из 3-4 стрелков\} для всех, кроме не очень метких и очень не метких?
  \item Покажи всех стрелков хуже чем не очень метких стрелков.
\end{enumerate}

\section{Схемы алгоритмов}

На рисунке \ref{img:brute} представлена схема алгоритма поиска полным перебором, на рисунке \ref{img:alg_all_search} представлена схема поиска всех значений словаря, удовлетворяющих условию.

\img{150mm}{brute}{Схема алгоритма поиска полным перебором.}
\clearpage
\img{150mm}{alg_all_search}{Схема алгоритма поиска всех значений в словаре, удовлетворяющих условию}
\clearpage

\section*{Вывод}

В данном разделе были рассмотрены структура словаря, на котором будут проводиться эксперименты, а также схемы алгоритмов поисков.
