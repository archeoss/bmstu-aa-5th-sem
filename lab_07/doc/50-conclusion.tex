\chapter*{Заключение}
\addcontentsline{toc}{chapter}{Заключение}
В данной лабораторной работе был рассмотрен алгоритм поиска в словаре объектов,
удовлетворяющих ограничению, заданному в вопросе на ограниченном естественном языке.
Было проведено анкетирование респондентов, на основе собранных данных построена функция принадлежности термам рассматриваемым объектам.
Из ограничений разработанной системы стоит выделить трудности с расширением возможного круга
вопросов, так как это потребует дополнительных условий, появятся дополнительные случаи, которые будет необходимо учитывать в контексте всего предложения и относительно остальных случаев.
В ходе выполнения лабораторной работы все поставленные задачи были
выполнены:
\begin{enumerate}[ 1{)}]
	\item формализован объект и его признак;
\item составлена анкета для заполнения респондентами;
\item проведено анкетирование респондентов;
\item построена функция принадлежности термам числовых значений признака, описываемого лингвистической переменной, на основе статистической обработки мнений респондентов, выступающих в роли экспертов;
\item описаны 3–5 типовых вопросов на русском языке, имеющих целью запрос на поиск в словаре;
\item описан алгоритм поиска в словаре объектов, удовлетворяющих ограничению, заданному в вопросе на ограниченном естественном языке;
\item описана структура данных словаря, хранящего наименование объектов
 \item реализован описанный алгоритм поиска в словаре;
 \item приведены примеры запросов пользователя и сформированной реализацией алгоритма поиска выборки объектов из словаря, используя составленные респондентами вопросы;
 \item дано заключение о применимости предложенного алгоритма и его ограничениях;
 \item описаны и обоснованы полученные результаты в виде отчета о выполненной лабораторной работе, выполненном как расчетно-пояснительная записка к работе.
\end{enumerate}
 Поставленная цель достигнута: описан и реализован алгоритм поиска по словарю при ограничении на значение признака, заданном при помощи лингвистической переменной.
