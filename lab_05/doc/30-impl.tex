\chapter{Технологическая часть}

В данном разделе приведены требования к программному обеспечению, средства реализации и сами реализации алгоритмов.

\section{Требования к программному обеспечению}
К программе предъявляется ряд условий:
\begin{itemize}
    \item[$-$] на вход ковейера подаётся массив задач, которые на нём нужно обработать;
	\item[$-$] на выходе - лог-запись, в которой записаны в упорядоченном по времени порядке события начала и конца обработки определённого задания на ленте.
    \item[$-$] ПО должно замерять время работы алгоритмов.
\end{itemize}

\section{Средства реализации}
Для реализации данной лабораторной работы необходимо установить следующее программное обеспечение:
\begin{itemize}
    \item[$-$] \href{https://www.rust-lang.org/}{Rust Programming Language v1.64.0} - язык программирования\cite{Rust}
    \item[$-$] \href{https://github.com/bheisler/criterion.rs}{Criterion.rs v0.4.0} - Средство визуализации данных
    \item[$-$] \href{https://www.latex-project.org/}{LaTeX} - система верстки документов
\end{itemize}

\section{Листинг кода}

В листингах \ref{lst:conveyor1} и \ref{lst:conveyor2} приведена реализация конвейера. В листингах \ref{lst:task1}, \ref{lst:task2}, \ref{lst:task3} и \ref{lst:task4} приведена реализация задачи.

\begin{lstinputlisting}[
	caption={Реализация конвейера, часть 1},
	label={lst:conveyor1},
	style={rust},
	linerange={1-39}
]{../src/dbscan/conveyor.rs}
\end{lstinputlisting}
\clearpage
\begin{lstinputlisting}[
	caption={Реализация конвейера, часть 2},
	label={lst:conveyor2},
	style={rust},
	linerange={41-55}
]{../src/dbscan/conveyor.rs}
\end{lstinputlisting}

\begin{lstinputlisting}[
	caption={Структура задачи, часть 1},
	label={lst:task1},
	style={rust},
    linerange={16-40}
]{../src/dbscan/task.rs}
\end{lstinputlisting}
\clearpage
\begin{lstinputlisting}[
	caption={Структура задачи, часть 2},
	label={lst:task2},
	style={rust},
    linerange={42-75}
]{../src/dbscan/task.rs}
\end{lstinputlisting}
\clearpage
\begin{lstinputlisting}[
	caption={Структура задачи, часть 3},
	label={lst:task3},
	style={rust},
    linerange={77-108}
]{../src/dbscan/task.rs}
\end{lstinputlisting}
\clearpage
\begin{lstinputlisting}[
	caption={Структура задачи, часть 4},
	label={lst:task4},
	style={rust},
    linerange={110-155}
]{../src/dbscan/task.rs}
\end{lstinputlisting}
% \begin{lstinputlisting}[
% 	caption={Структура задачи, часть 1},
% 	label={lst:task},
% 	style={rust},
%     linerange={77-96}
% ]{../src/dbscan/task.rs}
% \end{lstinputlisting}

% \begin{lstinputlisting}[
% 	caption={Дополнительные структуры},
% 	label={lst:add},
% 	style={rust}
% ]{../src/dbscan/additional_structs.rs}
% \end{lstinputlisting}

\section{Тестирование функций.}

Одним из способов протестировать работу DBSCAN является графическое отображение полученных кластеров. 

\img{100mm}{parallel0.png}{Тестирование работы DBSCAN}
Алгоритм успешно кластеризует данные.

\section*{Вывод}

Была разработана реализация конвейерных вычислений.
