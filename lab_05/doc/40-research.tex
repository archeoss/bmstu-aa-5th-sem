\chapter{Исследовательская часть}

Ниже приведены технические характеристики устройства, на котором было проведено тестирование ПО:

\begin{itemize}
    \item[$-$] Операционная система: Arch Linux \cite{arch} 64-бит.
    \item[$-$] Оперативная память: 16 Гб DDR4.
    \item[$-$] Процессор: 11th Gen Intel\textsuperscript{\tiny\textregistered} Core\textsuperscript{\tiny\texttrademark} i5-11320H @ 3.20 ГГц\cite{i5}.
\end{itemize}

Тестирование проводилось на ноутбуке, включенном в сеть электропитания. Во время тестирования ноутбук был нагружен только встроенными приложениями окружения, окружением, а также непосредственно системой тестирования.

\section{Сравнение параллельного и последовательного конвейера}

В рисунке \ref{img:plot} приведено сравнение времени выполнения линейного и параллельногоё конвейеров в зависимости от длины очереди (при количестве 4407 точек).

\clearpage

\img{100mm}{plot}{Время выполнения параллельного и последовательного конвейеров в зависимости от длины очереди}
% \begin{table}[h]
%     \caption{Время выполнения параллельного и последовательного конвейеров в зависимости от длины очереди}
%     \label{tab:comp}
%     \begin{center}
%         \begin{tabular}{ |c|c|c| }
%             \hline
%             \textbf{Длина очереди} & \textbf{Последовательный} & \textbf{Параллельный} \\
%             \hline
%             2 & 131913778 & 101065791 \\
%             \hline
%             4 & 264496449 & 169329387 \\
%             \hline
%             6 & 398330949 & 236552433 \\
%             \hline
%             8 & 528620040 & 299132268 \\
%             \hline
%             10 & 684301664 & 418758929 \\
%             \hline
%             12 & 766470705 & 434427701 \\
%             \hline
%             14 & 924169292 & 510653489 \\
%             \hline
%             16 & 1058106910 & 564733886 \\
%             \hline
%             18 & 1162946299 & 629284580 \\
%             \hline
%             20 & 1321981140 & 697166232 \\
%             \hline
%         \end{tabular}
%     \end{center}
% \end{table}
%

\section{Пример работы и анализ результата}

Пример работы программы приведен на рисунке \ref{img:queue}.
\clearpage
\img{200mm}{queue}{Пример работы программы}
\clearpage

\section*{Вывод}

В данном разделе были сравнены алгоритмы по времени. Выявлено,
что конвейерная обработка быстрее последовательной в среднем на 20\%
