\chapter{Аналитическая часть}

\section{Описание конвейерной обработки данных}

Конвейер\cite{conveyor} — способ организации вычислений, используемый в современных процессорах и контроллерах с целью повышения их производительности (увеличения числа инструкций, выполняемых в единицу времени — эксплуатация параллелизма на уровне инструкций), технология, используемая при разработке компьютеров и других цифровых электронных устройств.

Идея заключается в параллельном выполнении нескольких инструкций процессора. Сложные инструкции процессора представляются в виде последовательности более простых стадий. Вместо выполнения инструкций последовательно (ожидания завершения конца одной инструкции и перехода к следующей), следующая инструкция может выполняться через несколько стадий выполнения первой инструкции. Это позволяет управляющим цепям процессора получать инструкции со скоростью самой медленной стадии обработки, однако при этом намного быстрее, чем при выполнении эксклюзивной полной обработки каждой инструкции от начала до конца.

Многие современные процессоры управляются тактовым генератором. Процессор внутри состоит из логических элементов и ячеек памяти — триггеров. Когда приходит сигнал от тактового генератора, триггеры приобретают своё новое значение, и «логике» требуется некоторое время для декодирования новых значений. Затем приходит следующий сигнал от тактового генератора, триггеры принимают новые значения, и так далее. Разбивая последовательности логических элементов на более короткие и помещая триггеры между этими короткими последовательностями, уменьшают время, необходимое логике для обработки сигналов. В этом случае длительность одного такта процессора может быть соответственно уменьшена.

\section{Алгоритм DBSCAN}
Алгоритм, который был выбран для разложения на части конвейера -- DBSCAN (Density-Based Spatial Clustering of Applications with Noise)\cite{serial}\cite{https://doi.org/10.48550/arxiv.1912.06255}, алгоритм кластеризации, который основывается на понятии плотности. Он ищет группы точек, которые тесно связаны друг с другом и отделены друг от друга более редко встречающимися точками.

\section*{Вывод}
В данной работе стоит задача реализации конвейера для алгоритма DBSCAN.
