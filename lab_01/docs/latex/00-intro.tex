\chapter*{ВВЕДЕНИЕ}
\addcontentsline{toc}{chapter}{ВВЕДЕНИЕ}


\textbf{Расстояние Левенштейна} - минимальное количество операций вставки одного символа,
удаления одного символа и замены одного символа на другой, необходимых для превращения одной строки в другую.
\newline

Расстояние Левенштейна применяется в теории информации и компьютерной лингвистике для:

\begin{itemize}
    \item исправления ошибок в слове
    \item сравнения текстовых файлов утилитой diff
    \item в биоинформатике для сравнения генов, хромосом и белков
\end{itemize}

\textbf{Цель лабораторной работы:}
\begin{itemize}
    \item[$-$] Изучение и исследование особенностей задач динамического программирования.
\end{itemize}

\textbf{Задачи лабораторной работы:}
\begin{enumerate}
\item Изучить и реализовать алгоритмы Дамерау-Левенштейна.
\begin{enumerate}
    \item[$-$] Итеративный метод поиска Дамерау-Левенштейна;
    \item[$-$] Рекурсивный метод поиска Дамерау-Левенштейна;
    \item[$-$] Рекурсивный с кешированием метод поиска Дамерау-Левенштейна
\end{enumerate}
\item Создать ПО, реализующее алгоритмы, указанные в варианте.
\item Провести анализ затрат работы программы по времени и по памяти, выяснить влияющие на них характеристики.
    \item Создать отчёт, содержащий:
    \begin{enumerate}
        \item[$-$] схемы выбранных алгоритмов;
        \item[$-$] обоснование выбора технических средств;
        \item[$-$] результаты тестирования;
        \item[$-$] Выводы.
    \end{enumerate}
\end{enumerate}

\textbf{Для достижения поставленных целей и задач необходимо:}
\begin{enumerate}
    \item Изучить теоретические основы алгоритмов Дамерау-Левенштейна.
    \item Реализовать выше обозначенные алгоритмы.
    \item Провести экспериментальное исследование.
\end{enumerate}

\textbf{В ходе работы будут затронуты следующие темы:}
\begin{enumerate}
\item Динамическое программирование.
\item Алгоритмы поиска Дамерау-Левенштейна.
\item Оценка реализаций алгоритмов.
\end{enumerate}
