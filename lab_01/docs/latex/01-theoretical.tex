\chapter{Аналитическая часть}
Расстояние Левенштейна \cite{Lev} между двумя строками — это минимальное количество операций вставки,
удаления и замены, необходимых для превращения одной строки в другую.

Цены операций могут зависеть от вида операции (вставка (insert), удаление (delete), замена (replace)
и/или от участвующих в ней символов, отражая разную вероятность разных ошибок при вводе текста, и т. п. В общем случае:

\begin{itemize}
    \item $w(a,b)$ — цена замены символа $a$ на символ $b$.
    \item $w(\lambda,b)$ — цена вставки символа $b$.
    \item $w(a,\lambda)$ — цена удаления символа $a$.
\end{itemize}

Для решения задачи о редакционном расстоянии необходимо найти последовательность замен, минимизирующую суммарную цену.
Расстояние Левенштейна является частным случаем этой задачи при

\begin{itemize}
    \item $w(a,a)=0$.
    \item $w(a,b)=1, \medspace a \neq b$.
    \item $w(\lambda,b)=1$.
    \item $w(a,\lambda)=1$.
\end{itemize}

\section{Рекурсивный алгоритм нахождения расстояния Левенштейна}

Расстояние Левенштейна между двумя строками a и b может быть вычислено по формуле \eqref{eq:D},
где $|a|$ означает длину строки $a$; $a[i]$ — i-ый символ строки $a$ , функция $D(i, j)$ определена как:

\begin{equation}
    \label{eq:D}
    D(i, j) = \begin{cases}
                  0 &\text{i = 0, j = 0}\\
                  i &\text{j = 0, i > 0}\\
                  j &\text{i = 0, j > 0}\\
                  \min \lbrace \\
                  \qquad D(i, j-1) + 1\\
                  \qquad D(i-1, j) + 1 &\text{i > 0, j > 0}\\
                  \qquad D(i-1, j-1) + m(a[i], b[j])\\
                  \rbrace
    \end{cases},
\end{equation}

а функция $m(a, b)$ определена как:
\begin{equation}
    \label{eq:m}
    m(a, b) = \begin{cases}
                  0 &\text{если a = b,}\\
                  1 &\text{иначе}
    \end{cases}.
\end{equation}

Рекурсивный алгоритм реализует формулу \eqref{eq:D}.
Функция $D$ составлена из следующих соображений:
\begin{enumerate}
    \item Для перевода из пустой строки в пустую требуется ноль операций;
    \item Для перевода из пустой строки в строку $a$ требуется $|a|$ операций;
    \item Для перевода из строки $a$ в пустую требуется $|a|$ операций;
    \item Для перевода из строки $a$ в строку $b$ требуется выполнить последовательно некоторое количество операций (удаление, вставка, замена)
    в некоторой последовательности. Последовательность проведения любых двух операций можно поменять,
    порядок проведения операций не имеет никакого значения. Полагая,
    что $a', b'$  — строки $a$ и $b$ без последнего символа соответственно,
    цена преобразования из строки $a$ в строку $b$ может быть выражена как:
    \begin{enumerate}
        \item Сумма цены преобразования строки $a$ в $b$ и цены проведения операции удаления, которая необходима для преобразования $a'$ в $a$;
        \item Сумма цены преобразования строки $a$ в $b$  и цены проведения операции вставки, которая необходима для преобразования $b'$ в $b$;
        \item Сумма цены преобразования из $a'$ в $b'$ и операции замены, предполагая, что $a$ и $b$ оканчиваются разные символы;
        \item Цена преобразования из $a'$ в $b'$, предполагая, что $a$ и $b$ оканчиваются на один и тот же символ.
    \end{enumerate}
    Минимальной ценой преобразования будет минимальное
    значение приведенных вариантов.
\end{enumerate}

\section{Матричный алгоритм нахождения расстояния Левенштейна}

Прямая реализация формулы \eqref{eq:D} может быть малоэффективна по времени исполнения при больших $i, j$,
т. к. множество промежуточных значений $D(i, j)$ вычисляются заново множество раз подряд.
Для оптимизации нахождения расстояния Левенштейна можно использовать матрицу в целях хранения соответствующих промежуточных значений.
В таком случае алгоритм представляет собой построчное заполнение матрицы

$A_{|a|,|b|}$ значениями $D(i, j)$.

\section{Рекурсивный алгоритм нахождения расстояния Левенштейна с заполнением матрицы}

\label{sec:recmat}

Рекурсивный алгоритм заполнения можно оптимизировать по времени выполнения с использованием матричного алгоритма.
Суть данного метода заключается в параллельном заполнении матрицы при выполнении рекурсии.
В случае, если рекурсивный алгоритм выполняет прогон для данных, которые еще не были обработаны,
результат нахождения расстояния заносится в матрицу.
В случае, если обработанные ранее данные встречаются снова, для них расстояние не находится и алгоритм переходит к следующему шагу.

\section{Расстояние Дамерау — Левенштейна}

Расстояние Дамерау — Левенштейна может быть найдено по формуле \eqref{eq:d}, которая задана как

\begin{equation}
    \label{eq:d}
    d_{a,b}(i, j) = \begin{cases}
                        \max(i, j), &\text{если }\min(i, j) = 0,\\
                        \min \lbrace \\
                        \qquad d_{a,b}(i, j-1) + 1,\\
                        \qquad d_{a,b}(i-1, j) + 1,\\
                        \qquad d_{a,b}(i-1, j-1) + m(a[i], b[j]), &\text{иначе}\\
                        \qquad \left[ \begin{array}{cc}d_{a,b}(i-2, j-2) + 1, &\text{если }i,j > 1;\\
                        \qquad &\text{}a[i] = b[j-1]; \\
                        \qquad &\text{}b[j] = a[i-1]\\
                        \qquad \infty, & \text{иначе}\end{array}\right.\\
                        \rbrace
    \end{cases},
\end{equation}

Формула выводится по тем же соображениям, что и формула \eqref{eq:D}.
Как и в случае с рекурсивным методом, прямое применение этой формулы неэффективно по времени исполнения,
то аналогично методу из \eqref{sec:recmat} производится добавление матрицы для хранения промежуточных значений рекурсивной формулы.

\section*{Вывод}
В данном разделе были рассмотрены алгоритмы нахождения расстояния Левенштейна и Дамерау-Левенштейна,
который является модификаций первого, учитывающего возможность перестановки соседних символов.
Формулы Левенштейна и Дамерау — Левенштейна для рассчета расстояния между строками задаются рекурсивно,
а следовательно, алгоритмы могут быть реализованы рекурсивно или итерационно.

\clearpage