\chapter{Конструкторская часть}

В данной части будут рассмотрены схемы алгоритмов нахождения расстояния Дамерау - Левенштейна.
На рисунках \ref{img:dl_iterative} - \ref{img:dl_cacheRecursive} представлены рассматриваемые алгоритмы.

\section{Итеративный алгоритм}
\includeimage
{dl_iterative} % Имя файла без расширения (файл должен быть расположен в директории inc/img/)
{f} % Обтекание (без обтекания)
{h} % Положение рисунка (см. figure из пакета float)
{0.55\textwidth} % Ширина рисунка
{Схема итеративного алгоритма нахождения расстояния Дамерау-Левенштейна, первая часть} % Подпись рисунка
\clearpage

\includeimage
{dl_iterative_2} % Имя файла без расширения (файл должен быть расположен в директории inc/img/)
{f} % Обтекание (без обтекания)
{h} % Положение рисунка (см. figure из пакета float)
{0.75\textwidth} % Ширина рисунка
{Схема итеративного алгоритма нахождения расстояния Дамерау-Левенштейна, вторая часть} % Подпись рисунка
\clearpage

\section{Рекурсивный алгоритм}
\includeimage
{dl_recursive} % Имя файла без расширения (файл должен быть расположен в директории inc/img/)
{f} % Обтекание (без обтекания)
{h} % Положение рисунка (см. figure из пакета float)
{1\textwidth} % Ширина рисунка
{Схема рекурсивного алгоритма нахождения расстояния Дамерау-Левенштейна} % Подпись рисунка
\clearpage

\section{Рекурсивный алгоритм с кэшированием}
\includeimage
{dl_cacheRecursive} % Имя файла без расширения (файл должен быть расположен в директории inc/img/)
{f} % Обтекание (без обтекания)
{h} % Положение рисунка (см. figure из пакета float)
{0.9\textwidth} % Ширина рисунка
{Схема рекурсивного алгоритма с кэшем нахождения расстояния Дамерау-Левенштейна} % Подпись рисунка
\clearpage

\section*{Вывод}
На основе теоретических данных, полученные в аналитическом разделе были построены схемы исследуемых алгоритмов.

