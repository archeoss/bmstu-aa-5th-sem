\chapter{Технологическая часть}

В данном разделе приведены требования к программному обеспечению, средства реализации и листинги кода.

\section{Требования к программному обеспечению}
К программе предъявляется ряд условий:
\begin{itemize}
    \item[$-$] На вход подаются две строки в любой раскладке (в том числе и пустые);
    \item[$-$] На выход ПО должно выводить полученное расстояние и вспомогательные матрицы;
\end{itemize}

\section{Средства реализации}
Для реализации данной лабораторной работы необходимо установить следующее программное обеспечение:
\begin{itemize}
    \item \href{https://go.dev/}{Golang 1.19.1} - язык программирования
    \item \href{https://github.com/cep21/benchdraw}{Benchdraw 0.1.0} - Средство визуализации данных
    \item \href{https://www.latex-project.org/}{LaTeX} - система документооборота
\end{itemize}

\section{Листинги кода}
В листингах \ref{lst:ld-std}, \ref{lst:ld-rec} и \ref{lst:ld-cacheRec} приведены реализации алгоритмов
сортировки
\begin{itemize}
    \item[~---~] \("\)Итеративный метод поиска расстояния Дамерау-Левенштейна\("\);
    \item[~---~] \("\)Рекурсивный метод поиска расстояния Дамерау-Левенштейна\("\);
    \item[~---~] \("\)Рекурсивный с кешированием метод поиска расстояния Дамерау-Левенштейна\("\);
\end{itemize}
соответственно
\newpage
\subsection{Итеративный метод поиска расстояния Дамерау-Левенштейна}
\lstinputlisting[caption={Итеративный метод поиска расстояния Дамерау-Левенштейна}, label={lst:ld-std}, language=go, style=go,
firstline=6, lastline=36]{../../src/levenshtein/levenshtein.go}

\subsection{Рекурсивный метод поиска расстояния Дамерау-Левенштейна}
\lstinputlisting[caption={Рекурсивный метод поиска расстояния Дамерау-Левенштейна}, label={lst:ld-rec}, language=go, style=go,
firstline=37, lastline=72]{../../src/levenshtein/levenshtein.go}

\subsection{Рекурсивный с кешированием метод поиска расстояния Дамерау-Левенштейна}
\lstinputlisting[caption={Рекурсивный с кешированием метод поиска расстояния Дамерау-Левенштейна}, label={lst:ld-cacheRec}, language=go, style=go,
firstline=73, lastline=119]{../../src/levenshtein/levenshtein.go}

\section{Тестовые данные}

В таблице \ref{tab:o} приведены тестовые данные, на которых было протестированно разработанное ПО.

\begin{table}[h]
    \begin{center}
        \caption{Таблица тестовых данных}
        \label{tab:o}
        ""\newline
        \begin{tabular}{|c c c c c|}
            \hline
            № & Первое слово & Второе слово & Ожидаемый результат & Полученный результат \\ [0.8ex]
            \hline
            1 &  &  & 0 0 0 & 0 0 0\\
            \hline
            2 & kot & skat & 2 2 2 & 2 2 2\\
            \hline
            3 & kate & ktae & 1 1 1 & 1 1 1\\
            \hline
            4 & abacaba & aabcaab & 2 2 2 & 2 2 2\\
            \hline
            5 & sobaka & sboku & 3 3 3 & 3 3 3\\
            \hline
            6 & qwerty & queue & 4 4 4 & 4 4 4\\
            \hline
            7 & apple & aplpe & 1 1 1  & 1 1 1\\
            \hline
            8 &  & cat & 3 3 3 & 3 3 3\\
            \hline
            9 & parallels &  & 9 9 9 & 9 9 9\\
            \hline
        \end{tabular}
    \end{center}
\end{table}

\section*{Вывод}

В данном разделе были разработаны исходные программы трех алгоритмов:
вычисления расстояния Дамерау-Левенштейна рекурсивно, с заполнением матрицы и рекурсивно с заполнением матрицы.