\chapter{Аналитическая часть}

\section{Обзор алгоритмов поиска}

В данной работе были использованы два алгоритма поиска $-$ алгоритм полного перебора и муравьиный алгоритм.
\subsection{Алгоритм полного перебора}

\textbf{Алгоритм полного перебора} $-$ это алгоритм, предусматривающий перебор всех вариантов решения задачи. Он используется для поиска оптимального решения на небольших наборах данных.

\textbf{Алгоритм полного перебора для задачи поиска кратчайшего пути:}

Пронумеруем все города от 1 до $n$. Базовому городу присвоим номер n. 
Каждый тур по городам однозначно соответствует перестановке целых чисел $1, 2, ..., n-1$.


\begin{algorithm}
\caption{Алгоритм полного перебора для задачи поиска кратчайшего пути}
\label{alg:full_search}
\begin{algorithmic}[1]
\State \textbf{Вход:} матрица расстояний между городами
\State \textbf{Выход:} минимальное расстояние
\State \textbf{Переменные:} n $-$ количество городов, A $-$ матрица расстояний, x $-$ перестановка целых чисел, min $-$ минимальное расстояние
\State min $\leftarrow \infty$
\State p $\leftarrow 0$
\For {$k \leftarrow 1$ \textbf{to} $n$}
	\For {$i \leftarrow 1$ \textbf{to} $k! \cdot n$}
		\State p $\leftarrow$ p + 1
		\State x $\leftarrow$ перестановка $p$
		\State вычислить расстояние по перестановке x
		\If {расстояние $<$ min}
			\State min $\leftarrow$ расстояние
		\EndIf
	\EndFor
\EndFor
\State \textbf{возврат} min
\end{algorithmic}
\end{algorithm}

\textbf{Сложность алгоритма полного перебора:}

Временная сложность алгоритма полного перебора для поиска кратчайшего пути определяется как $O(n!)$ \cite{goodman}.

\subsection{Муравьиный алгоритм}

\textbf{Муравьиный алгоритм} $-$ один из эффективных алгоритмов для нахождения приближённых решений задачи поиска минимального пути, а также решения аналогичных задач поиска маршрутов на графах. Он базируется на модели поведения муравьёв, ищущих пути от колонии к источнику питания.

Алгоритм моделирует общее поведение муравьёв при поиске пути к источнику питания \cite{ulyanov}. 
Он имитирует поведение муравьёв, изменяя вероятность перехода из одной вершины в другую в зависимости от функции оценки и влияния окружающей среды.

\textbf{Муравьиный алгоритм для задачи поиска кратчайшего пути:}

Моделирование поведения муравьев связано с распределением феромона на тропе — ребре графа в задаче поиска минимального пути. При этом вероятность включения ребра в маршрут отдельного муравья пропорциональна количеству феромона на этом ребре, а количество откладываемого феромона пропорционально длине маршрута. Чем короче маршрут, тем больше феромона будет отложено на его ребрах, следовательно, большее количество муравьев будет включать его в синтез собственных маршрутов. Моделирование такого подхода, использующего только положительную обратную связь, приводит к преждевременной сходимости — большинство муравьев двигается по локально оптимальному маршруту. Избежать, этого можно, моделируя отрицательную обратную связь в виде испарения феромона. При этом если феромон испаряется быстро, то это приводит к потере памяти колонии и забыванию хороших решений, с другой стороны, большое время испарения может привести к получению устойчивого локально оптимального решения. Теперь, с учетом особенностей задачи поиска минимального пути, мы можем описать локальные правила поведения муравьев при выборе пути.

\begin{itemize}
	\item[$-$] муравьи имеют собственную «память». Поскольку каждый город может быть посещеи только один раз, у каждого муравья есть список уже посещенных городов --- список запретов. Обозначим через $J_{i,k}$ список городов, которые необходимо посетить муравью $k$, находящемуся в городе $i$;
	\item[$-$] муравьи обладают «зрением» --- видимость есть эвристическое желание посетить город $j$, если муравей находится в городе $i$. Будем считать, что видимость обратно пропорциональна расстоянию между городами $i$ и $j$ --- $D_{ij}$ 
	\begin{equation}
	\label{eq:vision}
	\eta_{ij} = \frac{1}{D_{ij}}
	\end{equation}
	\item[$-$] муравьи обладают «обонянием» — они могут улавливать след феромона, подтверждающий желание посетить город $j$ из города $i$, на основании опыта других муравьев. Количество феромона на ребре $(i,j)$ в момент времени $t$ обозначим через $\tau_{ij}(t)$.
\end{itemize}

На этом основании формулируется вероятностно-пропорциональное правило \ref{eq:rule}, определяющее вероятность перехода $k$-ого муравья из города $i$ в город $j$:

\begin{equation}
	\label{eq:rule}
	\begin{cases}
	P_{i,j,k}(t) = \frac{[\tau_{ij}(t)]^\alpha*[\eta_{ij}]^\beta}{\sum_{l\in J_{i,k}}^{}[\tau_{il}(t)]^\alpha * [\eta_{il}]^\beta}, & j \in J_{i,k};\\
	P_{i,j,k}(t) = 0, & j \notin J_{i,k},
	\end{cases}
\end{equation}

\section*{Вывод}

В данной секции были проанализированы два алгоритма поиска кратчайшего пути $-$ алгоритм полного перебора и муравьиный алгоритм. 
Алгоритм полного перебора имеет временную сложность $O(n!)$, а временная сложность муравьиного алгоритма зависит от количества итераций и выбранных параметров.

Алгоритм полного перебора используется для поиска оптимального решения на небольших наборах данных, а муравьиный алгоритм позволяет находить приближённые решения для больших наборов данных.
\clearpage
