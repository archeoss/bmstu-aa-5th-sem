\chapter{Технологическая часть}

В данном разделе приведены требования к программному обеспечению, средства реализации и сами реализации алгоритмов.

\section{Требования к программному обеспечению}
К программе предъявляется ряд условий:
\begin{itemize}
    \item[$-$] на вход подаётся матрица смежности графа;
    \item[$-$] на выходе программа выдаёт минимальную длину и путь, на котором данная длина достигнута;
    \item[$-$] ПО должно замерять время работы алгоритмов.
\end{itemize}

\section{Средства реализации}
Для реализации данной лабораторной работы необходимо установить следующее программное обеспечение:
\begin{itemize}
    \item \href{https://www.rust-lang.org/}{Rust Programming Language v1.64.0} - язык программирования
    \item \href{https://github.com/bheisler/criterion.rs}{Criterion.rs v0.4.0} - Средство визуализации данных
    \item \href{https://www.latex-project.org/}{LaTeX} - система документооборота
\end{itemize}

\section{Реализация алгоритмов}
В следующих листингах представлены следующие алгоритмы:
В листинге \ref{lst:brute} представлена реализация алгоритма полного перебора, в листинге \ref{lst:antsolver} представлена реализация муравьиного алгоритма. В листингах \ref{lst:config} и \ref{lst:ant} представлены конфигурационная структура и структура муравья с методами соотвественно.
\lstinputlisting[
        caption={Реализация полного перебора.},
        label={lst:brute},
        style={rust}
    ]{../../src/ants/brute_solver.rs}

\lstinputlisting[caption={Реализация муравьиного алгоритма.},
        label={lst:antsolver},
        style={rust}
    ]{../../src/ants/ant_solver.rs}
\lstinputlisting[
        caption={Конфигурационная структура.},
        label={lst:config},
        style={rust}
    ]{../../src/ants/ant_solver/config.rs}
\lstinputlisting[
        caption={Структура муравья и её методы.},
        label={lst:ant},
        style={rust}
    ]{../../src/ants/ant_solver/ant.rs}

\section{Тестирование функций.}

В таблице~\ref{tab:tests} приведены тесты для функции, реализующей алгоритм для решения задачи коммивояжера. Тесты пройдены успешно.

\begin{table}[h!]
    \begin{center}
        \caption{\label{tab:tests} Тестирование функций}
        \begin{tabular}{|c@{\hspace{7mm}}|c@{\hspace{7mm}}|c@{\hspace{7mm}}|c@{\hspace{7mm}}|}
            \hline
            Матрица смежности & Ожидаемый наименьший путь \\ \hline
            $\begin{pmatrix}
                0 &  9 &  12 &  45\\
                13 &  0 &  2 &  27\\
                13 &  8 &  0 &  21\\
                27 &  26 &  25 &  0
            \end{pmatrix}$ &
            32, [0, 1, 2, 3]\TBstrut \\ \hline
            $\begin{pmatrix}
                0 &  21 &  -1 &  9\\
                22 &  0 &  12 &  15\\
                -1 &  17 &  0 &  -1\\
                13 &  20 &  -1 &  0
            \end{pmatrix}$ &
            9, [0, 3]\TBstrut \\ \hline
            $\begin{pmatrix}
                0 &  24 &  8 &  28\\
                24 &  0 &  4 &  12\\
                14 &  6 &  0 &  26\\
                21 &  12 &  17 &  0
            \end{pmatrix}$ &
            26, [0, 2, 1, 3]\TBstrut \\ \hline
        \end{tabular}
    \end{center}
\end{table}

\section*{Вывод}
Спроектированные алгоритмы были реализованы и протестированы.
