\chapter*{ЗАКЛЮЧЕНИЕ}
\addcontentsline{toc}{chapter}{ЗАКЛЮЧЕНИЕ}

В рамках данной лабораторной работы лабораторной работы была достигнута её цель: изучен муравьиный алгоритм и приобретены навыки параметризации методов на примере муравьиного алгоритма.

Также выполнены следующие задачи:	
\begin{itemize}
	\item реализованны два алгоритма решения задачи поиска кратчайшего пути;
	\item замерено время выполнения алгоритмов;
	\item муравьиный алгоритм протестирован на разных переменных;
	\item сделаны выводы на основе проделанной работы;
\end{itemize}

Использовать муравьиный алгоритм для решения задачи коммивояжера выгодно (с точки зрения времени выполнения),
в сравнении с алгоритмом полного перебора, в случае если в анализируемом графе вершин больше либо равно 9. 
Так, например, при размере графа 11, муравьиный алгоритм работает быстрее чем алгоритм полного перебора в 15 раз. 
Стоит отметить, что муравьиный алгоритм не гарантирует что найденный путь будем оптимальным, так как он является эвристическим алгоритмом, в отличии от алгоритма полного перебора.

\indent Цели лабораторной работы по изучению и исследованию муравьиного алгоритма были достигнуты
