\chapter{Исследовательская часть}
Ниже приведены технические характеристики устройства, на котором было проведено тестирование ПО:

\begin{itemize}
    \item Операционная система: Arch Linux \cite{arch} 64-bit.
    \item Оперативная память: 16 Гб.
    \item Процессор: 11th Gen Intel\textsuperscript{\tiny\textregistered} Core\textsuperscript{\tiny\texttrademark} i5-11320H @ 3.20 ГГц\cite{i5}.
\end{itemize}

\section{Пример выполнения}
На рисунке \ref{img:example} приведен пример работы программы.
\img{80mm}{example}{Пример работы программы.}
\clearpage

\section{Время выполнения алгоритмов}
Алгоритмы тестировались при помощи инструментов замера времени предоставляемых библиотекой Criterion.rs\cite{Criterion}.
Пример функции по замеру времени приведен в листинге \ref{lst:benchmark}.
Количество повторов регулируется тестирующей системой самостоятельно, однако ввиду трудоемкости вычислений, количество повторов было ограничено до 25.
\lstinputlisting[language=Rust, style=rust, caption={Пример функции замера времени}, label={lst:benchmark} , firstline=9, lastline=43]{../../benches/bench.rs}

\newpage
График, показывающий время работы последовательного и параллельного алгоритмов в зависимости от количества потоков\newline

\img{100mm}{plot}{Замеры времени работы} \newpage


\section{Автоматическая параметризация}

В таблице \ref{tab:v5} приведена выборка результатов параметризации для матрицы смежности размером 10х10. Количество дней принято равным 100. Полным перебором был посчитан оптимальный путь -- он составил 130.

\begin{table}[H]

	\caption{Выборка из параметризации для матрицы размером $10x10$.}
	\label{tab:v5}
	\begin{center}

		\begin{tabular}{|c@{\hspace{7mm}}|c@{\hspace{7mm}}|c@{\hspace{7mm}}|c@{\hspace{7mm}}|c@{\hspace{7mm}}|c@{\hspace{7mm}}|c|}

			\hline
			$\alpha$        & $\beta$      & $\rho$      &Длина  & Разница & Путь \\

			\hline
0.9  & 0.1  & 0    & 6     & 0     & [0, 5, 9]\\
0.9  & 0.1  & 0.1  & 6     & 0     & [0, 5, 9]\\
0.9  & 0.1  & 0.2  & 6     & 0     & [0, 5, 9]\\
0.9  & 0.1  & 0.3  & 16    & 10    & [0, 5, 6, 2, 9]\\
\hline
0.9  & 0.1  & 0.4  & 6     & 0     & [0, 5, 9]\\
0.9  & 0.1  & 0.5  & 6     & 0     & [0, 5, 9]\\
0.9  & 0.1  & 0.6  & 6     & 0     & [0, 5, 9]\\
0.9  & 0.1  & 0.7  & 6     & 0     & [0, 5, 9]\\
\hline
0.9  & 0.1  & 0.8  & 6     & 0     & [0, 5, 9]\\
0.9  & 0.1  & 0.9  & 20    & 14    & [0, 6, 2, 9]\\
0.9  & 0.1  & 1    & 6     & 0     & [0, 5, 9]\\
1    & 0    & 0    & 18    & 12    & [0, 1, 7, 9]\\
\hline
1    & 0    & 0.1  & 6     & 0     & [0, 5, 9]\\
1    & 0    & 0.2  & 20    & 14    & [0, 6, 2, 9]\\
1    & 0    & 0.3  & 6     & 0     & [0, 5, 9]\\
1    & 0    & 0.4  & 6     & 0     & [0, 5, 9]\\
\hline
1    & 0    & 0.5  & 21    & 15    & [0, 1, 9]\\
1    & 0    & 0.6  & 18    & 12    & [0, 5, 6, 9]\\
1    & 0    & 0.7  & 6     & 0     & [0, 5, 9]\\
1    & 0    & 0.8  & 21    & 15    & [0, 1, 9]\\
		\hline
		\end{tabular}
	\end{center}
\end{table}

\begin{table}[H]
	\begin{center}
\begin{tabular}{|c@{\hspace{7mm}}|c@{\hspace{7mm}}|c@{\hspace{7mm}}|c@{\hspace{7mm}}|c@{\hspace{7mm}}|c@{\hspace{7mm}}|c|}
		\hline
		$\alpha$        & $\beta$      & $\rho$      &Длина  & Разница & Путь \\
			
			\hline
0.8  & 0.2  & 0.4  & 2100  & 0     & [0, 7, 4, 9]\\
0.8  & 0.2  & 0.5  & 2100  & 0     & [0, 7, 4, 9]\\
0.8  & 0.2  & 0.6  & 2127  & 27    & [0, 6, 5, 9]\\
0.8  & 0.2  & 0.7  & 2100  & 0     & [0, 7, 4, 9]\\
\hline
0.8  & 0.2  & 0.8  & 2100  & 0     & [0, 7, 4, 9]\\
0.8  & 0.2  & 0.9  & 2319  & 219   & [0, 9]\\
0.8  & 0.2  & 1    & 2100  & 0     & [0, 7, 4, 9]\\
0.9  & 0.1  & 0    & 2100  & 0     & [0, 7, 4, 9]\\
\hline
0.9  & 0.1  & 0.1  & 2100  & 0     & [0, 7, 4, 9]\\
0.9  & 0.1  & 0.2  & 2100  & 0     & [0, 7, 4, 9]\\
0.9  & 0.1  & 0.3  & 2100  & 0     & [0, 7, 4, 9]\\
0.9  & 0.1  & 0.4  & 2127  & 27    & [0, 6, 5, 9]\\
\hline
0.9  & 0.1  & 0.5  & 2319  & 219   & [0, 9]\\
0.9  & 0.1  & 0.6  & 2319  & 219   & [0, 9]\\
0.9  & 0.1  & 0.7  & 2319  & 219   & [0, 9]\\
0.9  & 0.1  & 0.8  & 2100  & 0     & [0, 7, 4, 9]\\
\hline
0.9  & 0.1  & 0.9  & 2319  & 219   & [0, 9]\\
0.9  & 0.1  & 1    & 2100  & 0     & [0, 7, 4, 9]\\
1    & 0    & 0    & 2100  & 0     & [0, 7, 4, 9]\\
1    & 0    & 0.1  & 2258  & 158   & [0, 6, 1, 9]\\
\hline
	\end{tabular}

\end{center}

\end{table}

\section*{Вывод}

При небольших размерах графа (от 3 до 7) алгоритм полного перебора выигрывает по времени у муравьиного. Например, при размере графа 5, полный перебор работает быстрее примерно в 57 раз. Однако, при увеличении размера графа (от 9 и выше), ситуация меняется в обратную сторону: муравьиный алгоритм начинает значительно выигрывать по времени у алгоритма полного перебора. 
Наиболее стабильные результаты автоматической параметризации получаются при наборе $\alpha = 0.1..0.5$, $\beta = 0.1..0.5$, $\rho = $ любое. При таких параметрах полученный результат не отличается более чем на 1 от эталонного, и, в около 75\% (на промежутке $\rho = 0.0..1.0$) случаев полученный результат совпадает с эталонным. Наиболее нестабильные результаты полученны при $\alpha = 1.0$, $\beta = 0.0$, $\rho = $ любое.


