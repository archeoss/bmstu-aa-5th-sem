\chapter*{ВВЕДЕНИЕ}
\addcontentsline{toc}{chapter}{ВВЕДЕНИЕ}

\textbf{Муравьиный алгоритм} $-$ один из эффективных полиномиальных алгоритмов для нахождения приближённых решений задачи коммивояжёра, а также решения аналогичных задач поиска маршрутов на графах. 
Суть подхода заключается в анализе и использовании модели поведения муравьёв, ищущих пути от колонии к источнику питания, и представляет собой метаэвристическую оптимизацию.
\clearpage
\section*{Цель лабораторной работы}
Целью данной лабораторной работы является изучение муравьиного алгоритма и приобретение навыков параметризации методов на примере муравьиного алгоритма.

\section*{Задачи лабораторной работы:}
В рамках выполнения работы необходимо решить следующие задачи:
	
\begin{itemize}
	\item решить задачу поиска кратчайшего пути при помощи алгоритма полного перебора и муравьиного алгоритма;
	\item замерить и сравнить время выполнения алгоритмов;
	\item протестировать муравьиный алгоритм на разных переменных;
	\item сделать выводы на основе проделанной работы.
\end{itemize}

\textbf{В ходе работы будут затронуты следующие темы:}


\begin{itemize}
	\item применение муравьиного алгоритма для решения задачи поиска кратчайшего пути;
	\item параметризация муравьиного алгоритма;
	\item алгоритм полного перебора;
	\item обработка графа.
\end{itemize}

